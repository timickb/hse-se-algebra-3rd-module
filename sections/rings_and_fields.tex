\documentclass[../main.tex]{subfiles}

\begin{document}
\stmnt{Опр} Пусть $K\neq \varnothing$ - множество, на котором заданы две бинарные операции: $+$ и $\cdot$
    такие, что:

    \void\numsec{1} $(K, +)$ - абелева группа;

    \void\numsec{2} $(K, \cdot)$ - полугруппа;

    \void\numsec{3} Умножение дистрибутивно по сложению:

    $$\forall a,b,c \in K: ((a+b)c = ac + bc) \wedge (c(a + b) = ca + cb)$$

    Тогда $(K, +, \cdot)$ - кольцо.

    \void\stmnt{Зам} $(K, +)$ - аддитивная группа кольца, $(K, \cdot)$ - мультипликативная
    полугруппа кольца.

    \void\stmnt{Опр} Подмножество $L\subseteq K$ называется подкольцом, если $\forall x,y\in L$:

    \void\numsec{1} $x - y\in L$ (т.е. L - подгруппа аддитивной группы кольца);

    \void\numsec{2} $x\cdot y\in L$ (т.е. L замкнуто относительно умножения).

    \void\stmnt{Примеры}

    \void\numsec{1} Числовые кольца: $(\Z, +, \cdot)$.

    $$ \Z \subset \mathbb{Q} \subset \R \subset \mathbb{C} $$

    \void\numsec{2} Полное матричное кольцо: $(M_n(\R), +, \cdot)$

    \void\stmnt{Опр} Если $(K, \cdot)$ - моноид, то говорят, что $(K, +, \cdot)$
    - кольцо с единицей.

    \void\numsec{3} Кольцо вычетов. Пусть $m\in \N, m > 1$. Тогда множество $m\Z\subset \Z$
    (подкольцо в $\Z$). $\Z$ разбивается на классы чисел, сравнимых по модулю $m$.

    Можно складывать и умножать эти классы.

    Пример: $(\Z_4, +, \cdot)$.

    \void\stmnt{Опр} Если $a\cdot b = 0$ при $a\neq 0, b\neq 0$ в кольце K, то $a$
    называется левым, а $b$ - правым делителем нуля.

    \void\stmnt{Опр} Если $\forall x,y \in K$: $x\cdot y = y\cdot x$, то кольцо называется
    коммутативным.

    \void\stmnt{Опр} Коммутативное кольцо с $1\neq 0$ и без делителей нуля называется
    \textbf{целостным кольцом}.

    \void\stmnt{Опр} $\varphi: K_1\rightarrow K_2$ - гомоморфизм колец, если $\forall a,b\in K_1$:

    \numsec{1} $\varphi(a + b) = \varphi(a) \oplus \varphi(b)$;

    \numsec{2} $\varphi(a\cdot b) = \varphi(a) * \varphi(b)$.

    \void\stmnt{Опр} Подмножество I кольца K называется двусторонним идеалом, если оно:

    \numsec{1} Является подгруппой по сложению;

    \numsec{2} $\forall a\in I, \forall r\in K$: $(r\cdot a\in I) \wedge (a\cdot r\in I)$.

    Пример: $m\Z$ - идеал в кольце $\Z$.

    \void\stmnt{Опр} Пусть K - коммутативное кольцо с единицей, тогда $K[x]$ - кольцо
    многочленов от $x$ с коэффициентами из K.

    Операции: сложение и умножение многочленов.

    \void\stmnt{Пример} $\R[x]$ - множество всех многочленов с вещественными коэффициентами.
    Это кольцо. Пример идеала: $I = \{ (x^2+1)\cdot f(x) \vert f(x)\in \R[x] \}$.

    \void\stmnt{Опр} Идеал I называется главным, если $\exists a\in K$:
    $I = \{r\cdot a \vert r\in K\}$. Говорят, что идеал I порожден $a$.

    \void\stmnt{Зам} В $\Z$ все идеалы главные.
    
    \void\stmnt{Опр} Элемент коммутативного кольца с единицей называется обратимым, если
    существует элемент $a^{-1}$: $a\cdot a^{-1} = a^{-1}\cdot a = 1$.

    \void\stmnt{Утв} Все обратимые элементы кольца K (с единицей) образуют
    группу $U(K)$ по умножению - это называется мультипликативной группой кольца.

    \void\stmnt{Опр} Поле P - это коммутативное кольцо с единицей ($1\neq 0$), в котором
    каждый элемент $a\neq 0$ обратим.

    \void\stmnt{Пример} $\mathbb{Q} ,\R, \mathbb{C}$ - поля.

    \void\stmnt{Опр} Подполе - подмножество поля... здесь все очевидно.

    \void\stmnt{Пример} $\Z_p$, где $p$ - простое, тоже является полем.

    \void\stmnt{Лемма} $\ker{\varphi}$, где $\varphi$ - гомоморфизм колец, всегда является
    идиалом в кольце $K_1$. ($\varphi: K_1\rightarrow K_2$).

    \void
    $\square$ Любой гомоморфизм колец является гомоморфизмом их аддитивных групп, следовательно
    $\ker{\varphi} является нормальной подгруппой в K_1$.

    Пусть $a\in \ker{\varphi}$, т.е. $\varphi(a) = 0$. Берем $a\cdot r$ и рассмотрим
    $\varphi(a\cdot r) = \varphi(a) * \varphi(r) = 0 * \varphi(r) = 0$.

    И аналогично $\varphi(r\cdot a) = \varphi(r)\cdot 0 = 0$. $(r\in K_1)$ $\blacksquare$

    \void\stmnt{Зам} Любой идеал является нормальной подгруппой в $(K, +) \Longrightarrow$
    можно рассмотреть факторгруппу $K/I$ (по сложению). Введем в этой факторгруппе умножение:
    $(a + I)(b + I) = a\cdot b + I$. (используется свойство идеала: $a\cdot I, b\cdot I\in I$).

    \void\stmnt{Опр} Множество $(K/I, +, \cdot)$ называется факторкольцом кольца K по идеалу I.

    \void\stmnt{Зам} Умножение корректно, так как $\forall a^{'}, b^{'}\in K$:

    $a + I = a^{'} + I, a^{'} = a + x, x\in I$;

    $b + I = b^{'} + I, b^{'} = b + y, y\in I$.

    $a^{'}\cdot b^{'} = (a + x)(b + y) + I = a\cdot b + a\cdot y + x\cdot b + I$.

    \void\stmnt{Пример} факторкольца

    $(\Z, +, \cdot)$ - кольцо, $(n\Z, +, \cdot)$ - идеал I в K. Тогда
    $\Z/n\Z = \Z_n$ - кольцо вычетов по модулю $n$.

    \void\stmnt{Зам} $Im(\varphi)$, где $\varphi$ - гомоморфизм колец, является подкольцом
    в $K_2$ ($\varphi: K_1\rightarrow K_2$).

    \void\stmnt{Теорема (о гомоморфизме колец)}

    Пусть $K_1, K_2$ - кольца и $\varphi: K_1\rightarrow K_2$ - их гомоморфизм.

    Тогда $K_1/\ker{\varphi} \cong Im(\varphi)$.

    \void
    $\square$ $\ker{\varphi}$ является идеалом (по Лемме) $\Rightarrow K_1/\ker{\varphi}$
    корректно определен.

    Рассмотрим отображение $\tau: K/\ker{\varphi}\rightarrow Im(\varphi)$,
    $\tau(a + i) = \varphi(a)$. Из доказательства о гомоморфизме групп следует, что $\tau$
    корректно определено и является гомоморфизмом группы по сложению. Остается проверить, что
    $\tau$ "уважает" умножение.

    $$\tau((a+I)\cdot(b+I)) = \tau(a\cdot b+I) = \varphi(ab) = \varphi(a)\cdot\varphi(b) =
    \tau(a+I) * \tau(b+I)$$

    Следовательно, $\tau$ - гомоморфизм колец и так как $\tau$ - биекция (из теоремы о гомоморфизме
    групп), то это и изоморфизм. $\blacksquare$

    \void\stmnt{Зам} Пусть K - целостное кольцо и $g$ - многочлен из $K[x]$, со старшим
    коэффициентом, обратимым в K. Тогда $\forall f\in K[x]$ существует единственная пара 
    $q(x), r(x)\in K[x]: f(x) = g(x)\cdot q(x) + r(x)$, причем $deg(r(x)) < deg(g(x))$, где
    $deg$ - степень многочлена.

    \void\stmnt{Алгоритм Евклида нахождения $gcd(a,b)$, где $a,b\in K$}

    $a(x) = b(x)\cdot q_1(x) + r_1(x)$, $deg(r_1(x)) < deg(b(x))$

    $b = q_2\cdot r_1 + r_2$, $deg(r_2) < deg(r_1)$

    .

    .

    .

    $r_{k-2} = q_k\cdot r_{k-1} + r_k$, $deg(r_k) < deg(r_1)$

    Последовательность степеней остатков строго убывающая. И так как она состоит
    из неотрицательных чисел, она обязательно должна оборваться, а это может произойти
    только засчет обнуления остатка: $r_{k-1} = q_{k-1}\cdot r_k$ (т.е. $r_{k+1} = 0$).

    Тогда $gcd(a, b)$ = $r_k$ - последний ненулевой остаток.

    \void\stmnt{Утв} В кольце многочленов $K[x]$ (K - целостное) $\forall a,b\in K[x]$ существует
    $gcd(a, b)$ и $\exists u(x),v(x)\in K[x]$: $gcd(a, b) = a(x)\cdot u(x) + b(x)\cdot v(x)$.

    \void\stmnt{Опр} (взаимной простоты многочленов)

    Два элемента кольца (многочленов) $a(x), b(x)$ - взаимнопростые, если 
    $\exists u(x),v(x)$: $a(x)\cdot u(x) + b(x)\cdot v(x) = 1$.

    \void\stmnt{Опр} Пусть Р - поле. Тогда характеристикой поля называется наименьшее
    $q\in\N$ такое, что: $1 + 1 + ... + 1 = 0$ ($q$ единиц). Если такого $q$ нет, то характеристика
    равна нулю. Обозначение: $char(P)$.

    \void\stmnt{Пример}

    \numsec{1} $char(\R) = char(\mathbb{C}) = char(\mathbb{Q}) = 0$;

    \numsec{2} $char(\Z_p) = p$ ($p$ - простое).

    \void\stmnt{Утв} $char(P)$ - либо ноль, либо простое число.

    \void
    $\square$ Пусть $p\neq 0 \Rightarrow p\geq 2 (1\neq 0)$.

    Если $p = mk$, где $1\leq m, k < p$, то $0 = 1+1+...+1$ ($mk$ штук)
    $= (1+1+...+1)(1+1+...+1)$ ($m$ и $k$ штук соответственно), так как $k$
    максимально, то обе скобки не нулевые, а значит, $m, k$ - делителя нуля, а их 
    не может быть в поле по определению.

    \void\stmnt{Опр} Пусть Р - поле. Рассмотрим множество рациональных функций
    (т.е. отношение двух многочленов) с коэффициентами из Р. Элементы этого множества
    - это дроби вида $\frac{f(x)}{g(x)}$, где $f,g\in P[x], g(x)\neq 0$. Это поле.

    Обозначение: $P(x)$ - поле частных.

    \void\stmnt{Зам} $char(\Z_p(x)) = p\neq 0$.

    \void\stmnt{Опр} Если $F\subset P$, то говорят, что Р является расширением поля F.

\end{document}
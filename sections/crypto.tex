\documentclass[../main.tex]{subfiles}

\begin{document}
\subsection{Задача дискретного логарифмирования.}
    Пусть $G$ - конечная группа и $g\in G$, причем $ord(g)$ достаточно большой.

    \stmnt{Задача} Для данного элемента $h\in \langle g\rangle$ найти $k: h = g^k$ (т.е. найти "логарифм").

    Оказывается, что дискретное логарифмирование - трудоемкая задача для обычного вычислительного устройства,
    а возведение в степень - нет.

    \subsection{Система Диффи-Хеллмана}

    Всем участникам переписки известна конечная группа G и элемент $g\in G$ (достаточно большого порядка).
    Участник A фиксирует натуральное число $a$ (оно секретно) и сообщает всем число $g^a$.

    Как создать общий ключ? У участника B есть секретное значение $g^b$. Тогда А возводит $g^b$ в
    степень $a$, а B возводит $g^a$ в степень $b$. В результате элемент $g^{ab}$ есть только у А и В,
    и они могут использовать его в качестве ключа для переписки.

    \subsection{Система Эль-Гамаля (1985)}

    Снова все участники знают G и $g\in G$. Участник А фиксирует натуральное число $a$ (оно секретно) и
    сообщает всем число $g^a$. Если другой участник хочет конфиденциально передать участнику А элемент
    $M\in G$, то он выбирает некоторое натуральное $k$ и сообщает всем пару чисел $(g^k, M\cdot(g^a)^k)$.
    По этим данным воостановить M может только А и он делает это так:

    $$ M = (M\cdot g^{ak})\cdot (g^k)^{\abs{G} - a} = M\cdot g^{ak}\cdot (g^k)^{\abs{G}}\cdot g^{-a\cdot k} =
    M\cdot (g^{\abs{G}})^k = M\cdot e^k = M$$

    \subsection{Система RSA (1997)}
    Криптографический алгоритм с открытым ключом. Применяется в TLS/SSL.

    \textbf{Общая идея}:

    \void\numsec{1} Выбираются 2 больших простых числа $p, q$ (обычно $>$ 2048 бит);

    \void\numsec{2} Вычисляется модуль $n = p\cdot q$;

    \void\numsec{3} Вычисляется функция Эйлера $\varphi = (p-1)(q-1)$;

    \void\numsec{4} Выбирается натуральное число $e$ (открытая экспонента), $1 < e < \varphi(n)$,
    $e$ взаимнопростое с $\varphi(n)$;

    \void\numsec{5} Вычисляется \textit{секретная экспонента} $d$ - обратое к $e$ по модулю $\varphi(n)$,
    то есть решение уравнения $x\cdot e = 1$ mod $\varphi(n)$ $\Longleftrightarrow x\cdot e = 1 + k\cdot\varphi(n)$.

    Вычисляется с помощью расширенного алгоритма Евклида.

    \void\numsec{6} Пара $(e, n)$ - открытый ключ; $(d, n)$ - закрытый ключ.

    \void\stmnt{Шифрование и дешифрование}

    Пусть В хочет отправить А сообщение М (это числа в диапазоне от 0 до $n - 1$).

    \void\textbf{Шифрование}: берем открытый ключ А: $(e, n)$ и сообщение М. Тогда
    $C = M^e$ mod $n$.

    \void\textbf{Дешифрование}: $M = C^d$ mod $n = M^{ed}$ mod $n$; $ed = 1 + k\cdot\varphi(n)$;
    $M^{1 + k\cdot\varphi(n)}$ mod $n$ = $M\cdot (M^{\varphi(n)})^k$ mod $n$ =
    $M\cdot 1^k$ mod $n$ = $M$.
\end{document}
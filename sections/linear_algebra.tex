\documentclass[../main.tex]{subfiles}

\begin{document}

Пусть F - поле, V - произвольное множество, на котором задано 2 операции:
\textbf{сложение} и \textbf{умножение на число} (т.е. на элемент из F). Это
означает, что $\forall x,y\in V$ $\exists$ элемент: $x + y\in V$ и $\forall \alpha\in F$ 
$\exists\alpha\cdot x\in V$.

Множество V называется векторным (или линейным) пространством, если 
$\forall x,y,z\in V$ и $\forall\alpha,\beta\in F$:

\void\numsec{1} $(x+y)+z = x+(y+z)$;

\void\numsec{2} Существует нейтральный элемент по сложению в $(V, +)$;

\void\numsec{3} Существует обратный элемент по сложению в $(V, +)$;

\void\numsec{4} $x + y = y + x$;

\void\numsec{5} $\forall x\in V$: $1\cdot x = x$ - нейтральность единицы из поля;

\void\numsec{6} Ассоциативность умножения на число: $\beta(\alpha x) = (\beta\alpha) x$;

\void\numsec{7} Дистрибутивность относительно сложения: $(\alpha +\beta)x = \alpha x + \beta x$;

\void\numsec{8} Дистрибутивность относительно сложения векторов.

\void\stmnt{Зам} Элементы линейного пространства называются векторами.

\void\stmnt{Примеры}

\void\numsec{0} $V_3$ - геометрические векторы;

\void\numsec{1} $F^n$, где F - поле,

$$F^n = \{ (x_1,...,x_n)^T \vert x_i\in F \}$$

Чаще всего это $\R^n$, а операции заданы покомпонентно.

$$ x =  \mx{x_1\\.\\.\\.\\x_n}, y = \mx{y_1\\.\\.\\.\\y_n} \Longrightarrow x+y = \mx{y_1+x_1\\.\\.\\.\\y_n+x_n} $$

$F^n$ - арифметическое пространство размерности $n$.

\void\numsec{2} $F^{\infty} = \{ (x_1,...,x_n,...) \vert x_i\in F, i\in\N \}$ - пространство
числовых последовательностей.

\void\numsec{3} Пусть $c[a, b]$ - множество функций непрерывных на отрезке $[a, b]$.

Операции:

$(f+g)(x) = g(x) + f(x)$ $\forall x\in [a, b]$;

$(\lambda f)(x) = \lambda\cdot f(x)$ $\forall x\in [a, b], \lambda\in \R$.

\void\numsec{4} Пусть $Ax = 0$ - однородная СЛАУ, $L$ - множество ее решений. Тогда
$\forall x,y\in L, \forall\lambda\in\R \Longrightarrow x+y\in L, \lambda\cdot x\in L
\Longrightarrow L$ - линейное пространство.

\subsection{Базис и размерность линейного пространства}

\void\stmnt{Опр} Базисом линейного пространства V называется упорядоченный набор векторов
$b_1,b_2,...,b_n$ такой, что:

\void\numsec(a) $b_1,...,b_n$ - линейно независимы;

\void\numsec{b} Любой вектор из V представляется в виде линейной комбинации $b_1,...b_n$, т.е.:

$\forall x\in V$: $x = x_1b_1 + ... + x_nb_n$.

При этом $x_1,...,x_n$ называются координатами вектора $x$ в базисе $b_1,...,b_n$.

\void\stmnt{Утв} Если $b_1,...,b_n$ - базис, то $\forall x\in V$ представляется в виде линейной комбинации
базисных векторов \textbf{единственным образом}.

\void
$\square$ Пусть $x = x_1b_1 + ... + x_nb_n = x^{'}_1b_1 + ... + x^{'}_nb_n \Longleftrightarrow$
$(x_1 - x^{'}_1)b_1 + ... + (x_n - x^{'}_n)b_n = 0$.

Векторы $b_1,...,b_n$ линейно независимы по определению базиса $\Rightarrow$

$\left\{
    \begin{matrix}
        x_1 - x^{'}_1 = 0\\
        .\\
        .\\
        .\\
        x_n - x^{'}_n = 0
    \end{matrix}
\right.$, т.е. $x_i = x^{'}_i \Longrightarrow$ разложение единственно. $\blacksquare$

\void\stmnt{Зам} При сложении векторов их координаты складываются, а при умножении на число -
умножаются на число!

\void\stmnt{Опр} Максимальное количество линейно независимых векторов в данном линейном
пространстве V называется размерностью этого линейного пространства. Обозначение: $dim(V)$.

\void\stmnt{Пример} $dim(V_3) = 3$, $dim(\R^n) = n$.

\void\stmnt{Зам} Рассмотрим линейное пространство V и фиксируем в нем базис $b_1,...,b_n$.
Тогда линейное пространство V изоморфно арифметическому линейному пространству $F^n$
(если линейное пространство над полем F).

$$x \longmapsto \mx{x_1\\.\\.\\.\\x_n}$$

Это изоморфизм: оно - биекция и "уважает" сложение и умножение на число.

\subsubsection{Переход к новому базису}

Пусть L = $n$-мерное линейное пространство.

$A = \{a_1,...,a_n\}, B = \{b_1,...,b_n\}$ - два базиса в L.

Разложим векторы базиса В по базису А:

$\left\{ 
    \begin{matrix}
        b_1 = t_{11}a_1 + t_{21}a_2 + ... + t_{n1}a_n\\
        .\\
        .\\
        .\\
        b_n = t_{1n}a_1 + t_{2n}a_2 + ... + t_{nn}a_n
    \end{matrix}
\right.$

\void\stmnt{Опр} Матрицей перехода от базиса А к базису В называется такая матрица:

$$ T_{A\rightarrow B} = \mx{t_{11} & t_{12} & . & . & . & t_{1n}\\.\\.\\.\\t_{n1} & t_{n2} & . & . & . & t_{nn}}$$

$$(b_1,...,b_n)_{1\times n} = (a_1,...,.a_n)\cdot T_{A\rightarrow B}$$

\fbox{$b = a\cdot T_{A\rightarrow B}$}

- матричная форма записи определения матрицы перехода, где 
$b = (b_1,...,b_n), a = (a_1,...,.a_n)$.

\void\stmnt{Зам} В любом базисе данного линейного пространства всегда одинаковое
количество векторов, равное $dim(V)$.

\void\stmnt{Зам} Матрица перехода всегда невырождена, так как векторы $b_i$ - линейно независимы
(по определению базиса) $\Rightarrow$ столбцы матрицы перехода тоже линейно независимы
(т.к. есть изоморфизм с $F^n$).

\void\stmnt{Утв} Пусть $x\in L$, $A, B$ - базисы в L.

$$x^a = \mx{x^a_1\\.\\.\\.\\x^a_n}, x^b = \mx{x^b_1\\.\\.\\.\\x^b_n}$$

- столбцы координат вектора $x$ в базисах $A$ и $B$ соответственно.

Тогда \fbox{$x^b = T^{-1}_{A\rightarrow B}\cdot x^a$}

\void
$\square$ Докажем, что $x^a = T_{A\rightarrow B}\cdot x^b$ (из невырожденности матрицы перехода следует нужная формула).

Пусть $x$ - исходный вектор.

$$ x = a\cdot x^a = (a_1,...,a_n)\cdot \mx{x^a_1\\.\\.\\.\\x^a_n} = x^a_1a_1 + ... + x^a_na_n = b\cdot x^b$$

$$ a\cdot x^a = a\cdot T_{A\rightarrow B}\cdot x^b \Longrightarrow$$
т.к. разложение по базису единственно, $x^a = T_{A\rightarrow B}\cdot x^b \Longrightarrow$
\fbox{$x^b = T^{-1}_{A\rightarrow B}\cdot x^a$} $\blacksquare$

\void\stmnt{Зам} $T_{B\rightarrow A} = T^{-1}_{A\rightarrow B}$

\void\stmnt{Зам} $x^b = T_{B\rightarrow A}\cdot x^a$

\void\stmnt{Утв} Пусть $\begin{matrix} A = \{a_1,...,a_n\} \\ B = \{b_1,...,b_n\} \\ C = \{c_1,...,c_n\}\end{matrix}$ - базисы в L.

Тогда $T_{A\rightarrow C} = T_{A\rightarrow B}\cdot T_{B\rightarrow C}$

\void $\square$ 

$\begin{matrix} c = b\cdot T_{B\rightarrow C} \\ b = a\cdot T_{A\rightarrow B} \\ c = a\cdot T_{A\rightarrow C} \end{matrix}$

Но $c = a\cdot T_{A\rightarrow B}\cdot T_{B\rightarrow C}$, а разложение по базису единственно, значит, $T_{A\rightarrow C} = T_{A\rightarrow B}\cdot T_{B\rightarrow C}$.
$\blacksquare$

\subsection{Подпространства}

\void\stmnt{Опр} Подмножество W линейного пространства V называется подпространством, если оно само является пространством относительно
операцией в V.

\void\stmnt{Зам} Для проверки того, что W является подпространством, достаточно проверять замкнутость относительно
сложения и унможения на число: $\forall x,y\in W, \forall\lambda\in F \Longrightarrow \left\{\begin{matrix}x+y\in W\\ \lambda x\in W\end{matrix}\right.$

\void\stmnt{Пример}

\void\numsec{1} $P[a, b]$ - многочлены на отрезке $[a, b]$ - подпространство в линейном 
пространстве $C[a, b]$.

\void\numsec{2} Пространство решений однородной СЛАУ $Ax = 0$ с $n$ незивестными -
подпространство в $\R^n$.

\void\numsec{3} В любом линейном пространстве есть подпространство $\{0\}$.

\void\stmnt{Опр} Множество $L(a_1,...,a_k) = \{\lambda_1 a_1 + ... + \lambda_n a_n \vert \lambda_i\in F\}$ -
множество всех линейных комбинаций векторов $a_1,...,a_k$. Называется \textbf{линейной оболочкой} набора $a_1,...,a_k$.

\void\stmnt{Утв} $\forall a_1,...,a_k \in V$: $L(a_1,...,a_k)$ всегда является подпространством $(k\in\N)$.

\void\stmnt{Пример} В $V_3$ с базисом $i, j, k$ $L(i, j)$ - плоскость $xOy$.

\void\stmnt{Опр} \textbf{Рангом} системы векторов $a_1,...,a_K$ в линейном пространстве называется размерность
их линейной оболочки.

$$Rg(a_1,...,a_k) = dim(L(a_1,...,a_k))$$

\void\stmnt{Утв} Ранг $(a_1,...,a_k)$ системы векторов линейного пространства V равен рангу матрицы,
составленной по столбцам из координат векторов $a_1,...,a_k$ в некотором базисе.

\void $\square$ Идея: при фиксированном базисе любое конечное линейное пространство изоморфно арифметическому
пространству, то есть пространству столбцов. А ранг системы столбцов равен рангу матрицы (из теоремы о ранге матрицы.)

$\blacksquare$

\void
Пусть $H_1, H_2$ - линейные подпространства в векторном пространстве L.

\void\stmnt{Утв} Множество $H_1\cap H_2$ является подпространством в L.

\void\stmnt{Зам} $H_1\cap H_2\neq \varnothing$, так как всегда есть 0.

\void\stmnt{Зам} $H_1\cup H_2$, вообще говоря, не является подпространством.

\void\stmnt{Пример} \textit{TODO}

\void\stmnt{Опр} $H_1 + H_2 = \{x_1 + x_2\vert x_1\in H_1, x_2\in H_2\}$ называется суммой подпространств.

\void\stmnt{Утв} $H_1 + H_2$ является подпространством.

\void\stmnt{Утв} Пусть $H_1, H_2$ - подпространства в L. Тогда $dim(H_1+H_2) = dim(H_1) + dim(H_2) - dim(H_1\cap H_2)$.

\void $\square$ Рассмотрим базис $H_1\cap H_2$. Дополним его до базиса в $H_1$ и до базиса в $H_2$.

Пусть $dim(H_1) = n, dim(H_2) = m, dim(H_1\cap H_2) = r.$



\end{document}
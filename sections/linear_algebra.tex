\documentclass[../main.tex]{subfiles}

\begin{document}

Пусть F - поле, V - произвольное множество, на котором задано 2 операции:
\textbf{сложение} и \textbf{умножение на число} (т.е. на элемент из F). Это
означает, что $\forall x,y\in V$ $\exists$ элемент: $x + y\in V$ и $\forall \alpha\in F$ 
$\exists\alpha\cdot x\in V$.

Множество V называется векторным (или линейным) пространством, если 
$\forall x,y,z\in V$ и $\forall\alpha,\beta\in F$:

\void\numsec{1} $(x+y)+z = x+(y+z)$;

\void\numsec{2} Существует нейтральный элемент по сложению в $(V, +)$;

\void\numsec{3} Существует обратный элемент по сложению в $(V, +)$;

\void\numsec{4} $x + y = y + x$;

\void\numsec{5} $\forall x\in V$: $1\cdot x = x$ - нейтральность единицы из поля;

\void\numsec{6} Ассоциативность умножения на число: $\beta(\alpha x) = (\beta\alpha) x$;

\void\numsec{7} Дистрибутивность относительно сложения: $(\alpha +\beta)x = \alpha x + \beta x$;

\void\numsec{8} Дистрибутивность относительно сложения векторов.

\void\stmnt{Зам} Элементы линейного пространства называются векторами.

\void\stmnt{Примеры}

\void\numsec{0} $V_3$ - геометрические векторы;

\void\numsec{1} $F^n$, где F - поле,

$$F^n = \{ (x_1,...,x_n)^T \vert x_i\in F \}$$

Чаще всего это $\R^n$, а операции заданы покомпонентно.

$$ x =  \mx{x_1\\.\\.\\.\\x_n}, y = \mx{y_1\\.\\.\\.\\y_n} \Longrightarrow x+y = \mx{y_1+x_1\\.\\.\\.\\y_n+x_n} $$

$F^n$ - арифметическое пространство размерности $n$.

\void\numsec{2} $F^{\infty} = \{ (x_1,...,x_n,...) \vert x_i\in F, i\in\N \}$ - пространство
числовых последовательностей.

\void\numsec{3} Пусть $c[a, b]$ - множество функций непрерывных на отрезке $[a, b]$.

Операции:

$(f+g)(x) = g(x) + f(x)$ $\forall x\in [a, b]$;

$(\lambda f)(x) = \lambda\cdot f(x)$ $\forall x\in [a, b], \lambda\in \R$.

\void\numsec{4} Пусть $Ax = 0$ - однородная СЛАУ, $L$ - множество ее решений. Тогда
$\forall x,y\in L, \forall\lambda\in\R \Longrightarrow x+y\in L, \lambda\cdot x\in L
\Longrightarrow L$ - линейное пространство.

\subsection{Базис и размерность линейного пространства}

\void\stmnt{Опр} Базисом линейного пространства V называется упорядоченный набор векторов
$b_1,b_2,...,b_n$ такой, что:

\void\numsec(a) $b_1,...,b_n$ - линейно независимы;

\void\numsec{b} Любой вектор из V представляется в виде линейной комбинации $b_1,...b_n$, т.е.:

$\forall x\in V$: $x = x_1b_1 + ... + x_nb_n$.

При этом $x_1,...,x_n$ называются координатами вектора $x$ в базисе $b_1,...,b_n$.

\void\stmnt{Утв} Если $b_1,...,b_n$ - базис, то $\forall x\in V$ представляется в виде линейной комбинации
базисных векторов \textbf{единственным образом}.

\void
$\square$ Пусть $x = x_1b_1 + ... + x_nb_n = x^{'}_1b_1 + ... + x^{'}_nb_n \Longleftrightarrow$
$(x_1 - x^{'}_1)b_1 + ... + (x_n - x^{'}_n)b_n = 0$.

Векторы $b_1,...,b_n$ линейно независимы по определению базиса $\Rightarrow$

$\left\{
    \begin{matrix}
        x_1 - x^{'}_1 = 0\\
        .\\
        .\\
        .\\
        x_n - x^{'}_n = 0
    \end{matrix}
\right.$, т.е. $x_i = x^{'}_i \Longrightarrow$ разложение единственно. $\blacksquare$

\void\stmnt{Зам} При сложении векторов их координаты складываются, а при умножении на число -
умножаются на число!

\void\stmnt{Опр} Максимальное количество линейно независимых векторов в данном линейном
пространстве V называется размерностью этого линейного пространства. Обозначение: $dim(V)$.

\void\stmnt{Пример} $dim(V_3) = 3$, $dim(\R^n) = n$.

\void\stmnt{Зам} Рассмотрим линейное пространство V и фиксируем в нем базис $b_1,...,b_n$.
Тогда линейное пространство V изоморфно арифметическому линейному пространству $F^n$
(если линейное пространство над полем F).

$$x \longmapsto \mx{x_1\\.\\.\\.\\x_n}$$

Это изоморфизм: оно - биекция и "уважает" сложение и умножение на число.

\subsubsection{Переход к новому базису}

Пусть L = $n$-мерное линейное пространство.

$A = \{a_1,...,a_n\}, B = \{b_1,...,b_n\}$ - два базиса в L.

Разложим векторы базиса В по базису А:

$\left\{ 
    \begin{matrix}
        b_1 = t_{11}a_1 + t_{21}a_2 + ... + t_{n1}a_n\\
        .\\
        .\\
        .\\
        b_n = t_{1n}a_1 + t_{2n}a_2 + ... + t_{nn}a_n
    \end{matrix}
\right.$

\void\stmnt{Опр} Матрицей перехода от базиса А к базису В называется такая матрица:

$$ T_{A\rightarrow B} = \mx{t_{11} & t_{12} & . & . & . & t_{1n}\\.\\.\\.\\t_{n1} & t_{n2} & . & . & . & t_{nn}}$$

\end{document}
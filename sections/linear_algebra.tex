\documentclass[../main.tex]{subfiles}

\begin{document}

Пусть F - поле, V - произвольное множество, на котором задано 2 операции:
\textbf{сложение} и \textbf{умножение на число} (т.е. на элемент из F). Это
означает, что $\forall x,y\in V$ $\exists$ элемент: $x + y\in V$ и $\forall \alpha\in F$ 
$\exists\alpha\cdot x\in V$.

Множество V называется векторным (или линейным) пространством, если 
$\forall x,y,z\in V$ и $\forall\alpha,\beta\in F$:

\void\numsec{1} $(x+y)+z = x+(y+z)$;

\void\numsec{2} Существует нейтральный элемент по сложению в $(V, +)$;

\void\numsec{3} Существует обратный элемент по сложению в $(V, +)$;

\void\numsec{4} $x + y = y + x$;

\void\numsec{5} $\forall x\in V$: $1\cdot x = x$ - нейтральность единицы из поля;

\void\numsec{6} Ассоциативность умножения на число: $\beta(\alpha x) = (\beta\alpha) x$;

\void\numsec{7} Дистрибутивность относительно сложения: $(\alpha +\beta)x = \alpha x + \beta x$;

\void\numsec{8} Дистрибутивность относительно сложения векторов.

\void\stmnt{Зам} Элементы линейного пространства называются векторами.

\void\stmnt{Примеры}

\void\numsec{0} $V_3$ - геометрические векторы;

\void\numsec{1} $F^n$, где F - поле,

$$F^n = \{ (x_1,...,x_n)^T \vert x_i\in F \}$$

Чаще всего это $\R^n$, а операции заданы покомпонентно.

$$ x =  \mx{x_1\\.\\.\\.\\x_n}, y = \mx{y_1\\.\\.\\.\\y_n} \Longrightarrow x+y = \mx{y_1+x_1\\.\\.\\.\\y_n+x_n} $$

$F^n$ - арифметическое пространство размерности $n$.

\void\numsec{2} $F^{\infty} = \{ (x_1,...,x_n,...) \vert x_i\in F, i\in\N \}$ - пространство
числовых последовательностей.

\void\numsec{3} Пусть $c[a, b]$ - множество функций непрерывных на отрезке $[a, b]$.

Операции:

$(f+g)(x) = g(x) + f(x)$ $\forall x\in [a, b]$;

$(\lambda f)(x) = \lambda\cdot f(x)$ $\forall x\in [a, b], \lambda\in \R$.

\void\numsec{4} Пусть $Ax = 0$ - однородная СЛАУ, $L$ - множество ее решений. Тогда
$\forall x,y\in L, \forall\lambda\in\R \Longrightarrow x+y\in L, \lambda\cdot x\in L
\Longrightarrow L$ - линейное пространство.

\subsection{Базис и размерность линейного пространства}

\void\stmnt{Опр} Базисом линейного пространства V называется упорядоченный набор векторов
$b_1,b_2,...,b_n$ такой, что:

\void\numsec(a) $b_1,...,b_n$ - линейно независимы;

\void\numsec{b} Любой вектор из V представляется в виде линейной комбинации $b_1,...b_n$, т.е.:

$\forall x\in V$: $x = x_1b_1 + ... + x_nb_n$.

При этом $x_1,...,x_n$ называются координатами вектора $x$ в базисе $b_1,...,b_n$.

\void\stmnt{Утв} Если $b_1,...,b_n$ - базис, то $\forall x\in V$ представляется в виде линейной комбинации
базисных векторов \textbf{единственным образом}.

\void
$\square$ Пусть $x = x_1b_1 + ... + x_nb_n = x^{'}_1b_1 + ... + x^{'}_nb_n \Longleftrightarrow$
$(x_1 - x^{'}_1)b_1 + ... + (x_n - x^{'}_n)b_n = 0$.

Векторы $b_1,...,b_n$ линейно независимы по определению базиса $\Rightarrow$

$\left\{
    \begin{matrix}
        x_1 - x^{'}_1 = 0\\
        .\\
        .\\
        .\\
        x_n - x^{'}_n = 0
    \end{matrix}
\right.$, т.е. $x_i = x^{'}_i \Longrightarrow$ разложение единственно. $\blacksquare$

\void\stmnt{Зам} При сложении векторов их координаты складываются, а при умножении на число -
умножаются на число!

\void\stmnt{Опр} Максимальное количество линейно независимых векторов в данном линейном
пространстве V называется размерностью этого линейного пространства. Обозначение: $dim(V)$.

\void\stmnt{Пример} $dim(V_3) = 3$, $dim(\R^n) = n$.

\void\stmnt{Зам} Рассмотрим линейное пространство V и фиксируем в нем базис $b_1,...,b_n$.
Тогда линейное пространство V изоморфно арифметическому линейному пространству $F^n$
(если линейное пространство над полем F).

$$x \longmapsto \mx{x_1\\.\\.\\.\\x_n}$$

Это изоморфизм: оно - биекция и "уважает" сложение и умножение на число.

\subsubsection{Переход к новому базису}

Пусть L = $n$-мерное линейное пространство.

$A = \{a_1,...,a_n\}, B = \{b_1,...,b_n\}$ - два базиса в L.

Разложим векторы базиса В по базису А:

$\left\{ 
    \begin{matrix}
        b_1 = t_{11}a_1 + t_{21}a_2 + ... + t_{n1}a_n\\
        .\\
        .\\
        .\\
        b_n = t_{1n}a_1 + t_{2n}a_2 + ... + t_{nn}a_n
    \end{matrix}
\right.$

\void\stmnt{Опр} Матрицей перехода от базиса А к базису В называется такая матрица:

$$ T_{A\rightarrow B} = \mx{t_{11} & t_{12} & . & . & . & t_{1n}\\.\\.\\.\\t_{n1} & t_{n2} & . & . & . & t_{nn}}$$

$$(b_1,...,b_n)_{1\times n} = (a_1,...,.a_n)\cdot T_{A\rightarrow B}$$

\fbox{$b = a\cdot T_{A\rightarrow B}$}

- матричная форма записи определения матрицы перехода, где 
$b = (b_1,...,b_n), a = (a_1,...,.a_n)$.

\void\stmnt{Зам} В любом базисе данного линейного пространства всегда одинаковое
количество векторов, равное $dim(V)$.

\void\stmnt{Зам} Матрица перехода всегда невырождена, так как векторы $b_i$ - линейно независимы
(по определению базиса) $\Rightarrow$ столбцы матрицы перехода тоже линейно независимы
(т.к. есть изоморфизм с $F^n$).

\void\stmnt{Утв} Пусть $x\in L$, $A, B$ - базисы в L.

$$x^a = \mx{x^a_1\\.\\.\\.\\x^a_n}, x^b = \mx{x^b_1\\.\\.\\.\\x^b_n}$$

- столбцы координат вектора $x$ в базисах $A$ и $B$ соответственно.

Тогда \fbox{$x^b = T^{-1}_{A\rightarrow B}\cdot x^a$}

\void
$\square$ Докажем, что $x^a = T_{A\rightarrow B}\cdot x^b$ (из невырожденности матрицы перехода следует нужная формула).

Пусть $x$ - исходный вектор.

$$ x = a\cdot x^a = (a_1,...,a_n)\cdot \mx{x^a_1\\.\\.\\.\\x^a_n} = x^a_1a_1 + ... + x^a_na_n = b\cdot x^b$$

$$ a\cdot x^a = a\cdot T_{A\rightarrow B}\cdot x^b \Longrightarrow$$
т.к. разложение по базису единственно, $x^a = T_{A\rightarrow B}\cdot x^b \Longrightarrow$
\fbox{$x^b = T^{-1}_{A\rightarrow B}\cdot x^a$} $\blacksquare$

\void\stmnt{Зам} $T_{B\rightarrow A} = T^{-1}_{A\rightarrow B}$

\void\stmnt{Зам} $x^b = T_{B\rightarrow A}\cdot x^a$

\void\stmnt{Утв} Пусть $\begin{matrix} A = \{a_1,...,a_n\} \\ B = \{b_1,...,b_n\} \\ C = \{c_1,...,c_n\}\end{matrix}$ - базисы в L.

Тогда $T_{A\rightarrow C} = T_{A\rightarrow B}\cdot T_{B\rightarrow C}$

\void $\square$ 

$\begin{matrix} c = b\cdot T_{B\rightarrow C} \\ b = a\cdot T_{A\rightarrow B} \\ c = a\cdot T_{A\rightarrow C} \end{matrix}$

Но $c = a\cdot T_{A\rightarrow B}\cdot T_{B\rightarrow C}$, а разложение по базису единственно, значит, $T_{A\rightarrow C} = T_{A\rightarrow B}\cdot T_{B\rightarrow C}$.
$\blacksquare$

\subsection{Подпространства}

\void\stmnt{Опр} Подмножество W линейного пространства V называется подпространством, если оно само является пространством относительно
операцией в V.

\void\stmnt{Зам} Для проверки того, что W является подпространством, достаточно проверять замкнутость относительно
сложения и унможения на число: $\forall x,y\in W, \forall\lambda\in F \Longrightarrow \left\{\begin{matrix}x+y\in W\\ \lambda x\in W\end{matrix}\right.$

\void\stmnt{Пример}

\void\numsec{1} $P[a, b]$ - многочлены на отрезке $[a, b]$ - подпространство в линейном 
пространстве $C[a, b]$.

\void\numsec{2} Пространство решений однородной СЛАУ $Ax = 0$ с $n$ незивестными -
подпространство в $\R^n$.

\void\numsec{3} В любом линейном пространстве есть подпространство $\{0\}$.

\void\stmnt{Опр} Множество $L(a_1,...,a_k) = \{\lambda_1 a_1 + ... + \lambda_n a_n \vert \lambda_i\in F\}$ -
множество всех линейных комбинаций векторов $a_1,...,a_k$. Называется \textbf{линейной оболочкой} набора $a_1,...,a_k$.

\void\stmnt{Утв} $\forall a_1,...,a_k \in V$: $L(a_1,...,a_k)$ всегда является подпространством $(k\in\N)$.

\void\stmnt{Пример} В $V_3$ с базисом $i, j, k$ $L(i, j)$ - плоскость $xOy$.

\void\stmnt{Опр} \textbf{Рангом} системы векторов $a_1,...,a_K$ в линейном пространстве называется размерность
их линейной оболочки.

$$Rg(a_1,...,a_k) = dim(L(a_1,...,a_k))$$

\void\stmnt{Утв} Ранг $(a_1,...,a_k)$ системы векторов линейного пространства V равен рангу матрицы,
составленной по столбцам из координат векторов $a_1,...,a_k$ в некотором базисе.

\void $\square$ Идея: при фиксированном базисе любое конечное линейное пространство изоморфно арифметическому
пространству, то есть пространству столбцов. А ранг системы столбцов равен рангу матрицы (из теоремы о ранге матрицы.)

$\blacksquare$

\void
Пусть $H_1, H_2$ - линейные подпространства в векторном пространстве L.

\void\stmnt{Утв} Множество $H_1\cap H_2$ является подпространством в L.

\void\stmnt{Зам} $H_1\cap H_2\neq \varnothing$, так как всегда есть 0.

\void\stmnt{Зам} $H_1\cup H_2$, вообще говоря, не является подпространством.

\void\stmnt{Пример} \textit{TODO}

\void\stmnt{Опр} $H_1 + H_2 = \{x_1 + x_2\vert x_1\in H_1, x_2\in H_2\}$ называется суммой подпространств.

\void\stmnt{Утв} $H_1 + H_2$ является подпространством.

\void\stmnt{Утв} Пусть $H_1, H_2$ - подпространства в L. Тогда $dim(H_1+H_2) = dim(H_1) + dim(H_2) - dim(H_1\cap H_2)$.

\void $\square$ Рассмотрим базис $H_1\cap H_2$. Дополним его до базиса в $H_1$ и до базиса в $H_2$.

Пусть $dim(H_1) = n, dim(H_2) = m, dim(H_1\cap H_2) = r.$

$$\ub{\ub{e_1,...,e_r}_{\text{базис в} H_1\cap H_2}, 
\ub{v_1,...,v_{n-r}}_{\text{доп. до базиса в} H_1},
\ub{w_1,...,w_{m-r}}_{\text{доп. до базиса в} H_2}}_{\text{Это базис в} H_1 + H_2}$$

... так как любой вектор из $H_1 + H_2$ может быть выражен через них и они линейно независимы.

$$dim(H_1+H_2) = r + (n - r) + (m - r) = n + m - r = dim(H_1) + dim(H_2) - dim(H_1\cap H_2)$$

$\blacksquare$

\void\stmnt{Опр} Сумма подпространств $H_1+H_2$ называется прямой и обозначается $H_1\oplus H_2$, если
$H_1\cap H_2 = \{0\}$, т.е. тривиально.

\void\stmnt{Следствие} $dim(H_1\oplus H_2) = dim(H_1) + dim(H_2)$.

\void\stmnt{Утв} $H_1 + H_2$ является прямой суммой $\Longleftrightarrow \forall x\in H_1+H_2$
единственным образом представляется в виде $x = x_1+x_2$, где $x_1\in H_1, x_2\in H_2$ (Критерий прямоты).

\void $\square$ \textbf{Достаточность:}

Дано: сумма прямая, т.е. $H_1\cap H_2 = \{0\}$.

Предположим, что есть 2 разные суммы: $x = x_1+x_2 = y_1+y_2$, $\left\{ \begin{matrix}x_1,y_1\in H_1\\ x_2,y_2\in H_2\end{matrix}\right.$

$x_1 - y_1 = y_2 - x_2 = 0$ (т.к. $H_1\cap H_2 = \{0\}$) $\Longrightarrow (x_1 = y_1) \wedge (x_2 = y_2)$.

\textbf{Необходимость}: Пусть представление единственно: $x = x_1+x_2$.

Если мы предположим, что $\exists x\neq 0$: $x\in H_1\cap H_2$, то
$\forall\alpha\in F$: $\alpha x\in H_1$ и $\alpha x\in H_2$. Тогда
$\forall\beta\in F$: $x = (1-\beta)x + \beta x \Longrightarrow$ представление не единственно - противоречие.
$\blacksquare$

\void
Пусть $L = H_1\oplus H_2$. Тогда $\forall x\in L$ может быть представлен единственным образом в виде
$x = y + z$, где $y\in H_1$, $z\in H_2$. 

\void\stmnt{Опр} $y$ называется проекцией $x$ на $H_1$ вдоль $H_2$, а $z$ - проекцией $x$ На
$H_2$ вдоль $H_1$.

$$y = pr_{H_1}x, z = pr_{H_2}x$$

\subsection{Билинейные формы}

Пусть V - линейное пространство над $\R$.

\void\stmnt{Опр} Функцию $b: V\times V \rightarrow \R$ называют билинейной формой, если $\forall x,y,z\in V, \forall\alpha,\beta\in\R$:

\void\numsec{1} $b(\alpha x + \beta y, z) = \alpha b(x, z) + \beta b(y, z)$;

\void\numsec{2} $b(x, \alpha y + \beta z) = \alpha b(x, y) + \beta b(x, z)$.

\void\stmnt{Пример} Любое скалярное произведение.

\void
Возьмем в V некоторый базис $e_1,...,e_n$. Тогда 
$$b(x,y) = b(x_1e_1 + ... + x_n e_n, y_1e_1 + ... + y_n e_n) = \sum_{i=1}^n\sum_{j=1}^n x_i\cdot y_i\cdot b(e_i,e_j)$$

$b(e_i,e_j)$ обозначим как $b_{ij}$.

\void\stmnt{Опр} Матрица $B = (b(e_i,e_j))_{n\times n}$ называется матрицей билинейное формы в базисе $e_1,...,e_n$.

\void\stmnt{Зам} Пусть $X = \mx{x_1\\.\\.\\.\\x_n}$ - столбец координат вектора $x$, $Y = \mx{y_1\\.\\.\\.\\y_n}$ -
столбец координат вектора $y$.

Тогда $b(x, y) = X^T\cdot B\cdot Y$.

\void\stmnt{Пример} $b(x, y) = x_1\cdot y_2 = \mx{x_1 & x_2}\cdot\mx{0 & 1\\ 0 & 0}\cdot\mx{y_1\\y_2}$

\void\stmnt{Зам} Между матрицами $n\times n$ и билинейными формами есть биекция (при фиксированном базисе).

\void\stmnt{Утв} Пусть $U$ - матрица перехода от базиса $e$ к базису $f$. Пусть $B_e$ - матрица билинейной формы
в базисе $e$, $B_f$ - аналогично в базисе $f$.

Тогда $B_f = U^T\cdot B_e\cdot U$.

\void $\square$ $b(x,y) = (x^e)^T\cdot B_e\cdot y^e = ...$ (здесь $x^e$ - столбец координат в базисе $e$).

$\left\{ \begin{matrix} x^e = U\cdot x^f\\ y^e = U\cdot y^f \end{matrix} \right.$ ($x^e$ - старые, а $x^f$ - новые)

$... = (U\cdot x^f)^T\cdot B_e\cdot (U\cdot y^f) = (x^f)^T\cdot \ub{U^T\cdot B_e\cdot U}_{B_f}\cdot y^f$

$\Longrightarrow B_f = U^T\cdot B_e\cdot U$ (т.к. $x, y$ - произвольные векторы и можно выбрать векторы базиса.)

$\blacksquare$

\subsection{Квадратичные формы}

\void\stmnt{Опр} Однородный многочлен от $n$ переменных, т.е.
$$Q(x) = \sum_{i=1}^n a_{ii} x_i^2 + 2\cdot\sum_{1\leqslant i\leqslant j\leqslant n} a_{ij} x_i x_j$$

$a_{ij}\in\R$ называют квадратичной формой.

\void
Рассмотрим $n$-мерное векторное пространство над $\R$. Выберем в нем базис $e_1,...,e_n$, и тогда
у произвольного вектора $x$ будет столбец координат $x^e = \mx{x_1\\.\\.\\.\\x_n} \Longrightarrow
Q(x)$ можно представить в виде $Q(x) = (x^e)^T\cdot A\cdot x^e$, где $A = (a_{ij})$ - коэффициенты из
формулы в определении.

\void
Заметим, что $Q(x): V\rightarrow \R$.

\void\stmnt{Пример} $Q(x) = x^2_1 + (2\cdot 4)x_1x_2 = \mx{x_1 & x_2 & x_3}\cdot\mx{1&4&0\\4&0&0\\0&0&0}\cdot\mx{x_1\\x_2\\x_3}$.

\void\stmnt{Зам} По любой билинейной форме можно построить квадратичную, взяв $Q(x) = b(x, x)$.

\void\stmnt{Опр} Билинейная форма называется симметрической, если $b(x,y) = b(y,x)$ и кососимметрической,
если $b(x,y) = -b(y, x)$.

\void\stmnt{Зам} По любой квадратичной форме можно построить симметрическую билинейную формулу.

$$ f(x,y) = \frac{1}{2} (Q(x+y) - Q(x) - Q(y)) $$

- это называется \textit{поляризация} квадратичной формы.

\void\stmnt{Зам} Достаточно взять в качестве матрицы билинейной формы матрицу квадратичной формы.

\void\stmnt{Утв} При переходе от базиса $e$ к базису $e^{'}$ одного и того же пространства $V$
матрица квадратичной формы меняется следующим образом:

$$A^{'} = S^T\cdot A\cdot S$$

Где $S$ - матрица перехода от $e$ к $e^{'}$, $A$ - матрица квадратичной формы в базисе $e$, $A^{'}$
- матрица квадратичной формы в $e^{'}$.

\void $\square$ $X = S\cdot X^{'}$

$$ Q(x) = \ub{X^T\cdot A\cdot X}_{\textbf{в базисе e}} = 
(S\cdot X^{'})^T\cdot A\cdot (S\cdot X^{'}) = (x^{'})^T\cdot S^T\cdot A\cdot S\cdot X^{'} =
(X^{'})^T\cdot A^{'}\cdot X^{'} \Longrightarrow$$
т.к. $x$ - произволен, $A^{'} = S^T A S$. $\blacksquare$

\void\stmnt{Опр} Рангом квадратной формы $Q(x) = x^T A x$ ($x$ - вектор)
называется ранг ее матрицы (т.е. $Rg(A)$).

\void\stmnt{Лемма} Пусть $A, S\in M_n(\R)$, $det(S)\neq 0$.

Тогда $Rg(AS) = RgA = Rg(SA)$ (т.е. умножение на невырожденную матрицу справа и слева не меняют
ранг матрицы А).

\void $\square$ $Rg(AS)\leqslant RgA$ (т.к. столбцы матрицы $AS$ - это линейные комбинации столбцов матрицы A).

Ранг $=$ максимальное количество линейно независимых столбцов (из теормы о ранге матрицы) $\Longrightarrow$
число линейно независимых столбцов не может вырасти и $Rg(AS)\leqslant RgA$.

$$ RgA = Rg(A\ub{SS^{-1}}_{E})\leqslant Rg(AS) \Longrightarrow RgA = Rg(AS)$$
$\blacksquare$

\void\stmnt{Утв} (об инвариантности ранга)

Пусть $Q$ - квадратичная форма на линейном пространстве.

$\begin{matrix} a = \{a_1,...,a_n\}\\ b=\{b_1,...,b_n\} \end{matrix}$ - базисы в V.

Пусть A - матрица $Q(x)$ в базисе $a$ и $B$ - аналогично в $b$.

Тогда $RgA = RgB$ (ранги матриц квадратичной формы)

\void $\square$ Мы знаем, что $B = S^TAS$, где $S = T_{a\rightarrow b}$ и она всегда невырождена.

$\Longrightarrow$ по лемме при умножении A на невырожденные матрицы $S$ и $S^T$ ее ранг не изменится, а
значит $RgB = RgA$. $\blacksquare$

\void\stmnt{Опр} Квадратичную форму Q(x) будем называть:

\void\numsec{1} Положительно определенной, если $\forall x\neq 0$: $Q(x) > 0$;

\void\numsec{2} Отрицательно определенной, если $\forall x\neq 0$: $Q(x) < 0$;

\void\numsec{3} Знакопеременной, если $\exists x,y\in V$: $Q(x) < 0 < Q(y)$;

\void\stmnt{Пример}

\void\numsec{1} $Q_1(x) = x^2_1 + x^2_2 + 5x^2_3$ - положительно определена;

\void\numsec{2} $Q_2(x) = x^2_1 - x^2_2 - x^2_3$ - знакопеременная.

\void
Пусть $Q(x) = x^TAx$, т.е. A - матрица квадратичной формы.

$$A = \mx{
    a_{11} & a_{12} & . & . & . & a_{1n}\\
    a_{12} & . & . & . & . & .\\
    .\\
    .\\
    .\\
    a_{1n} & . & . & . & . & a_{nn}
} 
\left. \begin{matrix}  
    \Delta_1 = a_{11}\\
    \Delta_2 = \dmx{a_{11}&a_{12}\\a_{12}&a_{22}}\\
    .\\
    .\\
    .\\
    \Delta_n = detA
\end{matrix}\right\} \text{последовательность главных угловых миноров}$$

\textit{Матрица A - симметрическая.}

\void\stmnt{Теорема} (критерий Сильвестра)

Квадратичная форма $Q(x)$ от $n$ переменных $x = (x_1,...,x_n)$ положительно определена
$\Longleftrightarrow \Delta_1 > 0, ..., \Delta_n > 0$.

\void\stmnt{Следствие} $Q(x)$ отрицательно определена $\Longleftrightarrow
\Delta_1 < 0, \Delta_2 > 0, \Delta_3 < 0, ..., (-1)^n\Delta_n > 0$, то есть
знаки главных угловых миноров чередуются, начиная с минуса.

\void\stmnt{Опр} Квадратичную форму $Q(x) = \alpha_1x_1^2 + \alpha_2x_2^2 + ... + \alpha_n x_n^2$,
$\alpha_i\in\R, i = \ol{1, n}$ (т.е. не имеющую попарных произведений элементов) называют квадратичной
формой \textbf{канонического вида}. Если $\alpha_i\in \{0,1,-1\}$, то канонический вид называют
нормальным.

\void\stmnt{Алгоритм} Метод Лагранжа приведения квадратичной формы к каноническому виду.

Метод состоит в последовательном выделении полных квадратов. На каждом шаге под квадрат должна
полностью уйти одна переменная. Если на каком-то этапе переменных в квадрате не осталось, но есть
выражение вида $C\cdot x_i\cdot x_j$, то делают замену переменных: 
$$x_i = x^{'}_i - x^{'}_j, x_j = x^{'}_i + x^{'}_j$$

(невырожденная замена) $\Longrightarrow$ не более чем за $2n$ шагов любая квадратичная форма
приводится к каноническому виду.

Если нужен нормальный вид, то выражение вида $C\cdot(x_i)^2$ заменяется на
$sgn(C)(\ub{\sqrt{\abs{C}}\cdot x_i^{'}}_{x^{''}_i})^2$

\void\stmnt{Пример} $Q(x) = x^2_1 - 4x_1x_2 = x^2_1 - 2x_12x_2 + (2x_2)^2-(2x_2)^2 =
(x_1-2x_2)^2 - (2x_2)^2 = ...$

Замена: $x^{'}_1 = x_i - 2x_2, x^{'}_2 = 2x_2$.

$... = x^{'2}_1 - x^{'2}_2$ - нормальный вид.

\void\stmnt{Зам} Канонический вид, к которому приводится квадратичная форма, определен
\textbf{неоднозначно}.

\void\stmnt{Теорема} (закон инерции квадратичных форм)

Для любых двух канонических видов

$$ Q_1(y_1,...,y_m) = \lambda_1y_1^2 + ... + \lambda_m y^2_m  (\lambda_i\neq 0, i = \ol{1, m})$$
$$ Q_2(z_1,...,z_k) = \mu_1z_1^2 + ... + \mu_k z^2_k  (\mu_j\neq 0, j = \ol{1, k})$$

одной и той же квадратичной формы выполнено:

\void\numsec{1} $m = k =$ ранг квадратичной формы;

\void\numsec{2} Количество положительных $\lambda_i =$ количеству положительных $\mu_j$ $(i_{+})$;

\void\numsec{3} Количество отрицательных $\lambda_i =$ количеству отрицательных $\mu_j$ $(i_{-})$.

\void Числа $i_{+}, i_{-}$ называют положительным и отрицательным индексами инерции соответственно.
(они являются инвариантами квадратичной формы)

\void\stmnt{Зам} $RgA = i_{+} + i_{-}$;

\void\stmnt{Зам} Иногда определеяют не $i_{+}$ и $i_{-}$, а их сумму
$i_{+} + i_{-} = r$ (ранг) и разность $i_{+} - i_{-} = s$ (сигнатура).

Но мы будем называть сигнатурой пару $(i_{+}, i_{-})$.

\subsection{Линейные отображения}
Пусть $V_1, V_2$ - два линейных (конечномерных) пространства над полем F.

\void\stmnt{Опр} Отображение $\varphi: V_1\rightarrow V_2$ называется линейным, если:

\numsec{1} $\forall x,y\in V_1$ $\varphi(x+y) = \varphi(x) + \varphi(y)$,

\numsec{2} $\forall x\in V_1$, $\forall\alpha\in F$ $\varphi(\alpha x) = \alpha\varphi(x)$.

\void\stmnt{Зам} Линейное отображение $\varphi$ - это гомоморфизм линейных пространств, т.е. $\varphi\in Hom(V_1,V_2)$.

\void
Пусть $e_1,...,e_n$ - базис в $V_1$, а $f_1,...,f_m$ - базис в $V_2$ ($dimV_1 = n, dimV_2 = m$).

Рассмотрим $\varphi(e_1),...,\varphi(e_n)\in V_2$ и разложим их по базису $f_1,...,f_m$ в $V_2$:

$$\begin{matrix}
    \varphi(e_1) = a_{11} f_1 + a_{21} f_2 + ... + a_{m1} f_m,\\
    .\\
    .\\
    .\\
    \varphi(e_n) = a_{1n} f_1 + a_{2n} f_2 + ... + a_{mn} f_m.
\end{matrix}$$

\void\stmnt{Опр} Матрица линейного отображения - это матрица

$$A_{ef} = \mx{
    a_{11} & . & . & . & . & a_{1n}\\
    a_{21} & . & . & . & . & a_{2n}\\
    . & . & . & . & . & .\\
    . &  &  &  &  & .\\
    . &  &  &  &  & .\\
    a_{m1} & . & . & . & . & a_{mn}\\
}$$

по столбцам которой стоят координаты образов векторов базиса $V_1$ в базисе $V_2$.

\void\stmnt{Опр} Линейное отображение пространства V в себя называется линейным оператором.
$$f: V\rightarrow V$$

\void\stmnt{Опр} Пусть $e_1,...,e_n$ - базис в $V_1$, а $f: V\rightarrow V$ - линейный оператор (л.о.), тогда матрица

$$A_{e} = \mx{
    a_{11} & . & . & . & . & a_{1n}\\
    a_{21} & . & . & . & . & a_{2n}\\
    . & . & . & . & . & .\\
    . &  &  &  &  & .\\
    . &  &  &  &  & .\\
    a_{n1} & . & . & . & . & a_{nn}\\
}$$

называется матрицей линейного оператора, если

$$\begin{matrix}
    \varphi(e_1) = a_{11} e_1 + a_{21} e_2 + ... + a_{n1} e_n,\\
    .\\
    .\\
    .\\
    \varphi(e_n) = a_{1n} e_1 + a_{2n} e_2 + ... + a_{nn} e_n.
\end{matrix}$$

\void\stmnt{Пример} $V_3$ - трехмерное геометрическое пространство, $L = L(i)$ - подпространство ($i$ - это 1-й вектор из стандартного базиса $i,j,k$).

$$
\begin{matrix}
    f(x) = pr_L^x\\
    f(i) = i = 1\cdot i + 0\cdot j + 0\cdot k\\
    f(j) = 0\\
    f(k) = 0
\end{matrix}
\Longrightarrow A_{\{i,j,k\}} =
\mx{1&0&0\\0&0&0\\0&0&0}
$$

\void\stmnt{Утв} (о том, что линейный оператор полностью задается матрицей при фиксированном базисе)

Пусть $f$ - линейным оператор в пространстве V, $e = \{e_1,...,e_n\}$ - базис в V. Пусть $x\in V$ и
$x^e = (x_1,...,x_n)^T$ - столбец координат вектора $x$ в базисе $e$.

Пусть $A_e$ - матрица линейного оператора $f$ в базисе $e$. Тогда $(f(x))^e = A_e\cdot x^e$.

\void $\square$ 
$$
f(x) = f(x_1e_1 + ... + x_n e_n) = x_1\cdot f(e_1) + ... + x_n\cdot f(e_n) =
x_1(a_{11}e_1 + ... + a_{n1}e_n) + ... + x_n(a_{1n}e_1 + ... + a_{nn}e_n) =$$
$$
(a_{11}x_1 + a_{12}x_2 + ... + a_{1n}x_n)e_1 + ... + (a_{n1}x_1+...+a_{nn}x_n)e_n
$$

$$
\Longrightarrow (f(x))^e =
\mx{
    a_{11}x_1 + a_{12}x_2 + ... + a_{1n}x_n\\
    .\\
    .\\
    .\\
    a_{n1}x_1 + a_{n2}x_2 + ... + a_{nn}x_n
} = A_e\cdot x^e \blacksquare
$$

\void\stmnt{Утв} Пусть $\varphi$ - линейное отображение из пространства $V_1$ в пространство $V_2$.
Пусть $A_{E_1E_2}$ - матрица линейного отображения в паре базисов: $E_1$ - базис в $V_1$, $E_2$ - базис в $V_2$.

Пусть даны две матрица перехода:

$T_1$ - матрица перехода от $E_1$ к $E_1^{'}$ в $V_1$,

$T_2$ - матрица перехода от $E_2$ к $E_2^{'}$ в $V_2$.

Тогда матрица линейного отображения в новой паре базисов $A_{E_1^{'}E_2^{'}} = T_2^{-1}\cdot A_{E_1E_2}\cdot T_1$.

\void
$\square$

$x^{E_1^{'}} = T_1^{-1}\cdot x^{E_1}$ - формула для замены координат в $V_1$,

$y^{E_2^{'}} = T_2^{-1}\cdot y^{E_2}$ - формула для замены координат в $V_2$.

Пусть $y$ - образ $x$ под действием $\varphi$ ($y = \varphi(x)$). Тогда по предыдущему утверждению

$y^{E_2} = A_{E_1E_2}\cdot x^{E_1}$ и $y^{E_2^{'}} = A_{E_1^{'}E_2^{'}}\cdot x^{E_1^{'}}$

$\Longrightarrow T_2^{-1} = y^{E_2} = A_{E_1^{'}E_2^{'}}\cdot T_1^{-1}\cdot x^{E_1} \Longrightarrow
y^{E_2} = \ub{ T_2\cdot A_{E_1^{'}E_2^{'}}\cdot T_1^{-1} }_{= A_{E_1E_2}}\cdot x^{E_1}$

$\Longrightarrow A_{E_1^{'}E_2^{'}} = T_2^{-1}\cdot A_{E_1E_2}\cdot T_1$ $\blacksquare$

\void\stmnt{Следствие} Пусть $\varphi$ - это $f$, то есть линейный оператор. Тогда $E_1 = E_2 = E$, $E_1^{'} = E_2^{'} = E^{'}$.

Тогда формула принимает вид $A_{E^{'}} = T^{-1}\cdot A_E\cdot T$.

\void\stmnt{Утв} С каждым линейным отображением $\varphi: V_1\rightarrow V_2$ связаны два подпространства:

$$Ker(\varphi)\subseteq V_1, Im(\varphi)\subseteq V_2$$

\void\stmnt{Опр} Ядром линейного отображения $\varphi$ называется

$$Ker\varphi = \{x\in V_1\vert \varphi(x) = 0\} = \varphi^{-1}(0)$$

\void\stmnt{Опр} Образом линейного отображения называется

$$Im\varphi = \{ x\in V_2\vert \exists y\in V_1: \varphi(y) = x \} = \varphi(V_1)$$

\void\stmnt{Зам} $Ker\varphi$ и $Im\varphi$ являются подпространствами.

\void\stmnt{Утв} Пусть $\varphi: V_1\rightarrow V_2$. Тогда $dim(Ker\varphi) + dim(Im\varphi) = m = dim(V_1)$.

\void
$\square$ Выберем базис в $V_1$: $e = \{e_1,...,e_m\}$. Тогда любой $x\in V_1$ можно представить в виде:

$$x = x_1e_1 + ... + x_m e_m\Longrightarrow \varphi(x) = x_1\cdot\varphi(e_1) + ... + x_m\cdot\varphi(e_m)$$

$Im\varphi = L(\varphi(e_1),...,\varphi(e_m))\Longrightarrow dim(Im\varphi) = RgA$.

Ядро линейного отображения записывается системой $Ax = 0\Longrightarrow dim(Ker\varphi)$ - это число элементов в ФСР
$Ax = 0$.

Но число элементов в ФСР - это $m - RgA$, то есть $m - RgA = dim(Ker\varphi)\Longrightarrow dim(Ker\varphi) + dim(Im\varphi) = m$.
$\blacksquare$

\void\stmnt{Зам} Пусть $\varphi$ - это линейный оператор, $\Longrightarrow Ker\varphi, Im\varphi$ - подпространства одного
линейного пространства V ($\varphi: V\rightarrow V$). Вообще говоря, $V\neq Ker\varphi\oplus Im f$, где
$\varphi$ - линейный оператор. ($dimV = dim(Ker\varphi) + dim(Im\varphi)$).

\void\stmnt{Пример} $D: g\mapsto g^{'}$ в $\R_n[x]$ (D - операция дифференцирования).

\void
$dim(\R_n[x]) = n+1$,

$ImD = \R_{n-1}[x]\Longrightarrow dim(ImD) = n$,

$KerD = L(1)$, $dim(KerD) = 1$.

$\ub{dim(ImD)}_{n} + \ub{dim(KerD)}_{1} = n+1$.

Но $KerD\cap ImD\neq \{0\}$ и $KerD+ImD = \R_{n-1}[x]\neq \R_n[x]$.

\subsubsection{Действия над линейными отображениями и их матрицами}

\void\stmnt{Опр} Пусть $A, B$ - линейные опраторы на пространстве V (над полем F), $\lambda\in F$, тогда:

$$(A+B)(x) = A(x)+B(x)$$
$$(\lambda A)(x) = \lambda\cdot A(x)$$
$$(A\cdot B)(x) = A(B(x))$$

\void\stmnt{Утв} Если фиксировать базис $e = \{e_1,...,e_n\}$, то

$$(A+B)^e = A^e + B^e$$
$$(\lambda A)^e = \lambda\cdot A^e$$
$$(A\cdot B)^e = A^e\cdot B^e$$

\void $\square$

$$((AB)x)^e = (A(Bx))^e = A_e\cdot(Bx)^e = A_e\cdot B_e\cdot x^e = (A_e\cdot B_e)\cdot x^e$$

$\blacksquare$

\void\textit{На этом лекции в 3-м модуле закончились.}

\end{document}
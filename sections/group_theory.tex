\documentclass[../main.tex]{subfiles}

\begin{document}
\stmnt{Утв} Пусть $G$ - группа и $g\in G$.

Тогда $\abs{\cycgr{g}} = ord(g)$, где $\abs{\cycgr{g}}$ - число элементов в циклической группе, порожденной элементом $g$.

\void
$\square$ Заметим, что если $g^k = g^s \Rightarrow g^{k-s} = e$ (так как $\exists g^{-1}$) $\Rightarrow$
порядок $g \leq k - s \Rightarrow$ если $g$ имеет бесконечный порядок, то все элементы $g^n$ $(n\in\Z)$
различны и $\Rightarrow \cycgr{g}$ содержит бесконечно много элементов. В бесконечном случае доказано.

\void
Если же $ord(g) = m$, то из минимальности $m\in\N \Rightarrow e=g^0, g=g^1, g^2,\dots,g^{m-1}$ попарно
различны. Покажем, что $\cycgr{g} = \{e, g, g^2,\dots,g^{m-1}\}$:

$\forall n\in\Z: n = m\cdot q + r$, где $0\leq r \le m \Rightarrow g^n = g^{mq+r} = (g^m)^q\cdot g^r =
e^q\cdot g^r = g^r$, где $0\leq r\le m \Rightarrow \cycgr{g} = \{e,g,g^2,\dots,g^{m-1}\}$
и $\abs{\cycgr{g}} = m = ord(g)$. $\blacksquare$

\void
\stmnt{Утв} Пусть $f: G\rightarrow F$ - гомоморфизм. Тогда $f$ - инъективен (т.е. является мономорфизмом)
$\Leftrightarrow \ker{f} = e_G$, где $e_G$ - нейтральный элемент в группе в $G$, а $\ker{f}$ в данном
случае является тривиальным \textbf{ядром} гомоморфизма.

\void
\stmnt{Опр} Ядром гомоморфизма $f: G\rightarrow F$ называется множество элементов группы G, которые переходят
в $e_F$ - нейтральный элемент во второй группе.

$$\ker{f} = \{g\in G\vert f(g) = e_F\}$$

\void
\stmnt{Зам} $\ker{f}$ никогда не является пустым множеством, так как по свойству гомоморфизма $f(e_G) = e_F$.

\void
$\square$

\fbox{$\Rightarrow$} Необходимость:

Дано: $\forall x_1 \neq x_2: f(x_1)\neq f(x_2)\Rightarrow f(e_G) = e_F$ 
(и для $x\in G$ $(x\neq e_G)$ $f(x)\neq f(e_G) = e_F$).

\fbox{$\Leftarrow$} Достаточность:

Дано: $\ker{f} = e_G$. Допустим, что $\exists x_1\neq x_2: f(x_1) = f(x_2)$.
Тогда $f(x_1\cdot x_2^{-1}) = f(x_1)\cdot f(x_2^{-1}) = f(x_1)\cdot(f(x_2))^{-1} = e_F$
$\Rightarrow x_1\cdot x_2^{-1} = e_G \Leftrightarrow x_1 = x_2$ - противоречие, значит,
$f$ инъективно. $\blacksquare$

\void
\stmnt{Опр} Таблица Кэли - это матрица из попарных произведений элментов из группы.

\void
\textbf{Примеры групп:}

\numsec{1} $D_n$ - группа диэдра - группа симметрий правильного $n$-угольника.
$$D_n = \{r, s\vert r^n = 1, s^2 = 1, s^{-1}rs = r^{-1}\}$$

\stmnt{Утв} $D_3\cong S_3$ ($S_3$ - группа подстановок).

\void
\numsec{2} $A_n\subset S_n $ ($A_n$ - все четные подстановки длины $n$).

$\abs{A_n} = \frac{n!}{2}$

\void
\numsec{3} Группа кватернионов:

$$Q_8 = \{\pm 1,\pm i,\pm, j,\pm k\vert (-1)^2 = 1, i^2 = j^2 = k^2 = -1 = ijk\}$$

\void
\textbf{Пример ядра:}

$$f: GL_n(\R)\rightarrow R^*$$

$f(A) = \det{A}$. Тогда $\ker f = \{A\vert \det{A} = 1\} = SL_n(\R)$ - специальная линейная группа.

\void
\stmnt{Утв} Любая подгруппа в $(\Z, +)$ имеет вид $k\Z$ (числа, кратные $k$) для некоторого $k\in \N\cup\{0\}$.

\void
$\square$ $k\Z$, очевидно, является подгруппой в $\Z$. Докажем, что других подгрупп не существует.

Если подгруппа $H = \{0\}$, то положим $k = 0$. Иначе: $k = min(H\cap\N)$. Тогда $k\Z\subseteq H$.

Если $a\in H$ и $a = qk + r$ $(0\leq r\le k) \Rightarrow r = a - kq$, где $a\in H$ и $kq\in H$, а значит,
$r = 0$ и $H = k\Z$ $\blacksquare$

\void
\stmnt{Опр} Пусть $G$ - группа и $H$ - ее подгруппа. Путь фиксирован $g\in G$. Тогда левым смежным классом
элемента $g$ по подгруппе $H$ называется множество:

$$ gH = \{g\cdot h\vert h\in H\} $$

Аналогично правым смежным классом является такое множество:
$$ Hg = \{h\cdot g\vert h\in H\} $$

\void
\stmnt{Лемма 1} $\forall g_1, g_2\in G$ либо $g_1H = g_2H$, либо $g_1H\cap g_2H = \varnothing$.

\void
$\square$ Если $g_1H\cap g_2H\neq\varnothing$, то $\exists h_1,h_2\in H: g_1h_1 = g_2h_2\Rightarrow
g_1 = g_2\cdot h_2\cdot h_1^{-1} \Rightarrow g_1H = g_2\cdot h_2\cdot h_1^{-1}H\subseteq H$.
А так как $h_2\cdot h_1^{-1}\in H$, то $g_1H\subseteq g_2H$. Аналогично существует
и обратное включение, а значит $g_1H = g_2H$. $\blacksquare$

\void
\stmnt{Лемма 2} $\abs{gH} = \abs{H}$ $\forall g\in G$ и для любой конечной подгруппы $H$.

\void
$\square$ $\abs{gH}\leq \abs{H}$. Если $gh_1 = gh_2 \Rightarrow g^{-1}gh_1 = g^{-1}gh_2
\Rightarrow h_1 = h_2$, то есть совпадений нет. $\blacksquare$

\void
\stmnt{Опр} Индексом подгруппы $H$ в группе $G$ называется количество левых смежных классов $G$ по подгруппе $H$.

Обозначение: $[G:H]$

\void
\stmnt{Теорема (Лагранжа)}

Пусть $G$ - конечная группа, $H$ - ее подгруппа, тогда $\abs{G} = \abs{H}\cdot [G:H]$.

\void
$\square$ Любой элемент группы лежит в своем левом смежном классе по $H$, и смежные классы не пересекаются
(по лемме 1) и любой из этих смежных классов содержит по $\abs{H}$ элементов (по лемме 2).
$\blacksquare$

\void
\stmnt{Следствие 1} Пусть $G$ - конечная группа и взят элемент $g\in G$. Тогда $ord(g)$ делит $\abs{G}$.

\void
$\square$ Возьмем $H = \cycgr{g}$. Мы знаем, что $\abs{\cycgr{g}} = ord(g)$ и
$\abs{G} = \abs{\cycgr{g}}\cdot [G:H]$, то есть $\abs{G}\vdots ord(g)$.
$\blacksquare$

\void
\stmnt{Следствие 2} Пусть $G$ - конечная группа, тогда $g^{\abs{G}} = e$.

\void
$\square$ Применим следствие 1: $\abs{G} = ord(g)\cdot s \Rightarrow g^{\abs{G}} = g^{ord(g)\cdot s} =
(g^{ord(g)})^s = e^s = e$.
$\blacksquare$

\void
\stmnt{Следствие 3} aka Малая Теорема Ферма:

Пусть $\ol{a}$ - ненулевой вычет попростому подулю $p$, тогда $\ol{a}^{p-1} = \ol{1}$,
то есть $a^{p-1}\equiv 1$ (mod $p$).

\void
\textit{$\ol{0}, \ol{1},\dots,\ol{p-1}$ - вычеты по модулю $p$, то есть остатки от деления
$m\in\Z$ на $p$.}

\void
$\square$ На самом деле это следствие 2 ($g^{\abs{G}} = e$), примененное к группе 
$\Z_p^* = \{ \Z_p\backslash \{0\}, \cdot \}$, где $Z_p$ - множество всех вычетов по модулю $p$.
$\abs{Z_p^*} = p - 1 \Rightarrow \ol{a}^{\abs{Z_p^*}} = e$.
$\blacksquare$

\void
\stmnt{Зам} Точно так же можно было рассмотреть и правые смежные классы. Но число левых смежных классов
равно числу правых и равно $\frac{\abs{G}}{\abs{H}}$ (по теореме Лагранжа).

\void
\stmnt{Опр} Подгруппа H группы G называется нормальной, если $gH = Hg$ $\forall g\in G$.

Обозначение: $H \normin G$.

\void
\stmnt{Опр} Группа называется простой, если она не имеет собственных нормальных подгрупп.

\void
\stmnt{Зам} Циклическая группа является простой тогда и только тогда, когда ее порядок
$p$ конечен и является простым числом.

\void
\stmnt{Опр} \textbf{Факторгруппа}. Пусть $H$ - нормальная подгруппа в группе $G$. Тогда $G/H$ - множество
левых смежных классов по H с операцией умножения: $(g_1H)\cdot(g_2H) = g_1g_2H$.

\void
\stmnt{Зам} Операция умножения смежных классов ассоциативна, есть нейтральный элемент: $eH = H$ и
для любого $gH$ есть обратный элемент $g^{-1}H$. Нормальность подгруппы нужна для корректности
определения умножения.

$g_1g_2H = g_1h_1g_2h_2H$, $g_1h_1g_2h_2 = g_1g_2(g_2^{-1}h_1g_2)h_2$.

\void
\stmnt{Утв} Пусть $H\subseteq G$ - подгруппа, тогда следующие 3 условия эквивалентны:

\void\numsec{1} $H\normin G$;

\void\numsec{2} $gHg^{-1}\subseteq H$ $\forall g\in G$;

\void\numsec{3} $\forall g\in G$: $gHg^{-1} = H$.

\void
\stmnt{Утв} $h\normin G \Leftrightarrow H = \ker{f}$, где $f$ - любой гомоморфизм.

\void
$\square$ Достаточность:

Если $f: G\rightarrow F$, то $\forall z\in \ker{f}$: $f(g^{-1}zg)$ $\forall g\in G$.

$\forall g\in G$ покажем, что $g^{-1}zg \in \ker{f}$: $f(g^{-1}zg) = f(g^{-1})\cdot f(z)\cdot f(g) =
(f(g))^{-1}\cdot f(z)\cdot f(g) = (f(g))^{-1}\cdot f(g) = e_F \Rightarrow$ по определению ядра
$g^{-1}zg\in \ker{f} \Rightarrow$ ядро - нормальная подгруппа.

\void
Необходимость:

Дано: H - нормальная поднруппа. Доказать: $\exists f$ - гомоморфизм, такой что $H = \ker{f}$.

В роли $f$ может выступать естественный гомоморфизм $e: G\rightarrow G/H$. Он существует,
так как $H\normin G$ и $G/H$ корректно определена. $\ker{f}$ - это множество всех элементов,
которые перешли в $eH = H$ - исходная нормальная подгруппа.

\void
\stmnt{Теорема (о гомоморфизме групп)}

Пусть $f: G\rightarrow F$ - гомоморфизм групп. Тогда образ $f$ = $Im(f) = \{ a\in F\vert \exists g\in G: f(g) = a\}$
изоморфен (как группа) факторгруппе $g/\ker{f}$, т.е.
\fbox{$G/\ker{f} \cong Im(f)$}

\void\stmnt{Зам} $Im(f)$ всегда является подгруппой в $F$.

\void
$\square$ Рассмотрим отображение $\tau: G/\ker{f}\rightarrow F$, заданное формулой
$\tau(g\ker{f}) = f(g)\in Im(f)$, где $g\ker{f}$ - смежный класс по $H = \ker{f}$.

Докажем, что $\tau$ - и есть искомый гомоморфизм. Проверим корректность, т.е. покажем,
что от выбора представителя смежного класса ничего не зависит.

$$\forall h_1,h_2\in \ker{f}: f(gh_1) = f(g)\cdot f(h_1) = f(g)\cdot e_F = f(g) = 
f(g)\cdot f(h_2) = f(gh_2)$$

$\Rightarrow \tau$ определен корректно.

Отображение $\tau$ сюръективно ($\tau: G/\ker{f}\rightarrow Im(f)$). Покажем,
что оно инъективно:

$f(g) = e_F \Longleftrightarrow g\in \ker{f} = H$, т.е. ядро гомоморфизма состоит только из
нейтрального элемента в факторгруппе. Воспользуемся критерием инъективности: $\tau$
инъективно $\Longleftrightarrow \ker{f}$ тривиально (состоит из $e\ker{f}$) $\Longrightarrow \tau$
- биекция.

Остается проверить, что $\tau$ - гомоморфизм:

$$\tau((g_1\ker{f})\cdot (g_2\cdot\ker{f})) = \tau(g_1g_2\ker{f}) = f(g_1g_2) = f(g_1)\cdot f(g_2)
= \tau(g_1\ker{f})\cdot \tau(g_2\ker{f})$$

$\Longrightarrow \tau$ - биективный гомоморфизм, т.е. изоморфизм. $\blacksquare$

\void
\stmnt{Примеры}

\void\numsec{1} $f: GL_n(\R) \rightarrow \R^*$ (сопоставляет матрице ее определитель)

$$\ker(det) = SL_n(\R) = \{A\vert det(A) = 1\}$$

$\Rightarrow$ по теореме $GL_n(\R)/SL_n(\R) \cong R^*$.

\void\numsec{2} $f: \Z \rightarrow \Z_n$

$\ker{f} = n\Z$ - числа, кратные $n$. $\Z/\Z_n \cong \Z_n$.

\void
\stmnt{Опр} Прямым произведением двух групп $G_1$ и $G_2$ называется их декартово
произведение как множеств с покомпонентным умножением.

$$(x_1, y_1) \circ (x_2, y_2) = (x_1 * x_2, y_1 \cdot y_2)$$
* - операция в $G_1$, $\cdot$ - операция в $G_2$.
Обозначение: $G_1\times G_2$.

\void
\stmnt{Утв} Если $G_1, G_2$ - группы, то $G_1\times G_2$ - тоже группа.

\void
\stmnt{Зам} Существует 5 неизоморфных между собой групп порядка 8. Из них 3
абелевых: $\Z_8, \Z_2\times\Z_4$ и 2 неабелевых: $D_4, Q_8$.

\void
\stmnt{Теорема Кэли}

Любая конечная группа порядка $n$ изоморфна некоторой подгруппе группы $S_n$.

\void
$\square$ Пусть $\abs{G} = n$ и $\forall a\in G$ рассмотрим отображение
$L_a: G\rightarrow G$, определенное формулой: $L_a(g) = a\cdot g$. Покажем, что
$L_a$ - биекция. Пусть $e, g_2, g_3,...,g_n$ - элементы группы. Тогда
$a\cdot e, a\cdot g_2,...,a\cdot g_n$ - те же самые элементы группы, но в
другом порядке $\Longrightarrow L_a$ - перестановка элементов группы.
При этом относительно операции композиции отображений $\forall L_a: \exists (L_a)^{-1} =
L_a^{-1}, \exists$ нейтральный элемент $id = L_e$ и есть ассоциативность. Из ассоциативности
в $G$ $L_{ab}(g) = (ab)\cdot g = a\cdot (bg) = L_a(L_b\cdot g) \Leftrightarrow L_{ab} = L_a\circ L_b$
$\Rightarrow$ множество $L_e, L_{g_2},...,L_{g_n}$ образуют подгруппу H в группе S(G) всех
биективных отображений G на себя, т.е. в $S_n$. Искомый изоморфизм:
$a \mapsto L_a\in H\subseteq S_n$. $\blacksquare$ 

\void\stmnt{Опр} Автоморфизм - это изоморфизм из G в G. Пример: $L_a$.

\void\stmnt{Зам} Множество всех автоморфизмов обозначается $Aut(G)$.

\void\stmnt{Опр} Внутренним автоморфизмом называют отображение $I_a: g\mapsto a\cdot g\cdot a^{-1}$.

\void\stmnt{Зам} Все внутренние автоморфизмы тоже образуют группу
$I_{nn}(G)\subseteq Aut(G)$.

\void\stmnt{Зам} Если G - абелева, то $I_{nn}(G) - \{e\}$.

\void\stmnt{Опр} \textbf{Центр группы} G - это подмножество $Z(G) = \{a\in G\vert a\cdot b = b\cdot a \forall b\in G\}$, то
есть множество элементов, которые коммутируют со всеми.

\void\stmnt{Зам} G - абелева $\Longleftrightarrow Z(G) = G$.

\void\stmnt{Утв} Z(G) всегда является нормальной подгруппой в G.

\void
$\square$ Покажем, что Z(G) является подгруппой. Для того, чтобы H было подгруппой:
$\forall a,b\in H: ab^{-1}\in H$. Проверим:

\void\numsec{1} $e\in H$ - берем $b = a$, $a\cdot a^{-1} = e\in H$;

\numsec{2} $a\cdot b\in H$ - берем $b = b^{-1} \Longrightarrow a\cdot b\in H$;

\numsec{3} $a^{-1}\in H$ - берем $a = e, b = a \Longrightarrow a^{-1}\in H$.

Проверим, что $\forall a, b\in Z(G)$ $a\cdot b^{-1}\in Z(G)$: $a\cdot b^{-1}g = a\cdot b^{-1}\cdot(g^{-1})^{-1} =
a(g^{-1}\cdot b)^{-1} = a(b\cdot g^{-1})^{-1} = a\cdot g\cdot b^{-1} = g\cdot a\cdot b^{-1}$, ч.т.д.
$\Longrightarrow Z(G)$ - подгруппа.

Эта подгруппа - нормальная, так как элементы коммутируют с любыми и $gZ(G) = Z(G)g$. $\blacksquare$

\void\stmnt{Утв} $G/Z(G) \cong I_{nn}(G)$.

\void\stmnt{Зам} Если G - абелева, то $G/Z(G) \cong \{e\}$.

\void
$\square$ Факторгруппа $G/Z(G)$ корректно определена, т.к. $Z(G)\normin G$.
Рассмотрим отображение $f: G\rightarrow Aut(G)$, заданное формулой:
$f: g\mapsto \varphi_g(h) = g\cdot h\cdot g^{-1}$.

Тогда $Im(f) = I_{nn}(G)$ по определению и $\ker{f} = Z(G)$, т.к. $ghg^{-1} = h \Longleftrightarrow
gh = hg$ - тогда это верно для элементов, коммутирующих с любыми, т.е. $\in Z(G)$. $\blacksquare$
\end{document}
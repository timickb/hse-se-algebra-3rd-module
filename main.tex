\documentclass[a4paper,11pt]{article}

\usepackage[utf8]{inputenc}
\usepackage[english,russian]{babel}
\usepackage{titling}
\usepackage{titlesec}
\usepackage{amsfonts,amsmath,amssymb,amsthm,mathtools}
\usepackage{icomma}
\usepackage{cmap}
\usepackage{mathtext}


\newcommand{\stmnt}[1]{\fbox{\textbf{#1}}}
\newcommand{\void}{\vspace{\baselineskip}}
\newcommand{\R}{\mathbb{R}}
\newcommand{\N}{\mathbb{N}}
\newcommand{\Z}{\mathbb{Z}}
\newcommand{\abs}[1]{\vert #1\vert}
\newcommand{\numsec}[1]{\textbf{#1).}}
\newcommand{\normin}{\vartriangleleft}
\newcommand{\mx}[1]{\begin{pmatrix} #1\end{pmatrix}}

\setlength{\parindent}{0em}
\setlength{\droptitle}{-7em}

\usepackage[left=1cm,right=1cm,
    top=1.5cm,bottom=1.5cm,bindingoffset=0cm]{geometry}
\usepackage{subfiles}

\title{Алгебра на ФКН ПИ}
\author{Общий конспект всех лекций за 3 модуль}

\begin{document}
    \maketitle

    \section{Теория групп (окончание)}
    
    \subfile{sections/group_theory.tex}

    \newpage
    \section{Применение конечных групп к задачам криптографии с открытым ключом}

    \subfile{sections/crypto.tex}

    \newpage
    \section{Кольца и поля}

    \subfile{sections/rings_and_fields.tex}

    \newpage
    \section{Линейные пространства}

    \subfile{sections/linear_algebra.tex}

\end{document}
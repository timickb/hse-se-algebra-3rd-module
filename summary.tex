\documentclass[a4paper,11pt]{article}

\usepackage[utf8]{inputenc}
\usepackage[english,russian]{babel}
\usepackage{titling}
\usepackage{titlesec}
\usepackage{amsfonts,amsmath,amssymb,amsthm,mathtools}
\usepackage{icomma}
\usepackage{cmap}
\usepackage{mathtext}

\titleformat{\section}
  {\normalfont\normalsize}{\fbox{}}{1em}{}

\newcommand{\stmnt}[1]{\fbox{\textbf{#1}}}
\newcommand{\void}{\vspace{\baselineskip}}
\newcommand{\R}{\mathbb{R}}
\newcommand{\N}{\mathbb{N}}
\newcommand{\Z}{\mathbb{Z}}
\newcommand{\abs}[1]{\vert #1\vert}
\newcommand{\numsec}[1]{\textbf{#1).}}

\setlength{\parindent}{0em}
\setlength{\droptitle}{-7em}

\usepackage[left=2cm,right=2cm,
    top=2cm,bottom=2cm,bindingoffset=0cm]{geometry}

\title{Алгебра на ФКН ПИ}
\author{Общий конспект всех лекций за 3 модуль}

\begin{document}
    \maketitle

    \stmnt{Утв} Пусть $G$ - группа и $g\in G$.

    Тогда $\abs{\langle g\rangle} = ord(g)$, где $\abs{\langle g\rangle}$ - число элементов в циклической группе, порожденной элементом $g$.

    \void
    $\square$ Заметим, что если $g^k = g^s \Rightarrow g^{k-s} = e$ (так как $\exists g^{-1}$) $\Rightarrow$
    порядок $g \leq k - s \Rightarrow$ если $g$ имеет бесконечный порядок, то все элементы $g^n$ $(n\in\Z)$
    различны и $\Rightarrow \langle g\rangle$ содержит бесконечно много элементов. В бесконечном случае доказано.
    
    \void
    Если же $ord(g) = m$, то из минимальности $m\in\N \Rightarrow e=g^0, g=g^1, g^2,\dots,g^{m-1}$ попарно
    различны. Покажем, что $\langle g\rangle = \{e, g, g^2,\dots,g^{m-1}\}$:

    $\forall n\in\Z: n = m\cdot q + r$, где $0\leq r \le m \Rightarrow g^n = g^{mq+r} = (g^m)^q\cdot g^r =
    e^q\cdot g^r = g^r$, где $0\leq r\le m \Rightarrow \langle g\rangle = \{e,g,g^2,\dots,g^{m-1}\}$
    и $\abs{\langle g\rangle} = m = ord(g)$. $\blacksquare$

    \void
    \stmnt{Утв} Пусть $f: G\rightarrow F$ - гомоморфизм. Тогда $f$ - инъективен (т.е. является мономорфизмом)
    $\Leftrightarrow \ker{f} = e_G$, где $e_G$ - нейтральный элемент в группе в $G$, а $\ker{f}$ в данном
    случае является тривиальным \textbf{ядром} гомоморфизма.

    \void
    \stmnt{Опр} Ядром гомоморфизма $f: G\rightarrow F$ называется множество элементов группы G, которые переходят
    в $e_F$ - нейтральный элемент во второй группе.

    $$\ker{f} = \{g\in G\vert f(g) = e_F\}$$

    \void
    \stmnt{Зам} $\ker{f}$ никогда не является пустым множеством, так как по свойству гомоморфизма $f(e_G) = e_F$.

    \void
    $\square$

    \fbox{$\Rightarrow$} Необходимость:

    Дано: $\forall x_1 \neq x_2: f(x_1)\neq f(x_2)\Rightarrow f(e_G) = e_F$ 
    (и для $x\in G$ $(x\neq e_G)$ $f(x)\neq f(e_G) = e_F$).

    \fbox{$\Leftarrow$} Достаточность:

    Дано: $\ker{f} = e_G$. Допустим, что $\exists x_1\neq x_2: f(x_1) = f(x_2)$.
    Тогда $f(x_1\cdot x_2^{-1}) = f(x_1)\cdot f(x_2^{-1}) = f(x_1)\cdot(f(x_2))^{-1} = e_F$
    $\Rightarrow x_1\cdot x_2^{-1} = e_G \Leftrightarrow x_1 = x_2$ - противоречие, значит,
    $f$ инъективно. $\blacksquare$

    \void
    \stmnt{Опр} Таблица Кэли - это матрица из попарных произведений элментов из группы.

    \void
    \textbf{Примеры групп:}

    \numsec{1} $D_n$ - группа диэдра - группа симметрий правильного $n$-угольника.
    $$D_n = \{r, s\vert r^n = 1, s^2 = 1, s^{-1}rs = r^{-1}\}$$

    \stmnt{Утв} $D_3\cong S_3$ ($S_3$ - группа подстановок).

    \void
    \numsec{2} $A_n\subset S_n $ ($A_n$ - все четные подстановки длины $n$).

    $\abs{A_n} = \frac{n!}{2}$

    \void
    \numsec{3} Группа кватернионов:

    $$Q_8 = \{\pm 1,\pm i,\pm, j,\pm k\vert (-1)^2 = 1, i^2 = j^2 = k^2 = -1 = ijk\}$$

    \void
    \textbf{Пример ядра:}

    $$f: GL_n(\R)\rightarrow R^*$$

    $f(A) = \det{A}$. Тогда $\ker f = \{A\vert \det{A} = 1\} = SL_n(\R)$ - специальная линейная группа.

    \void
    \stmnt{Утв} Любая подгруппа в $(\Z, +)$ имеет вид $k\Z$ (числа, кратные $k$) для некоторого $k\in \N\cup\{0\}$.

    \void
    $\square$ $k\Z$, очевидно, является подгруппой в $\Z$. Докажем, что других подгрупп не существует.

    Если подгруппа $H = \{0\}$, то положим $k = 0$. Иначе: $k = min(H\cap\N)$. Тогда $k\Z\subseteq H$.

    Если $a\in H$ и $a = qk + r$ $(0\leq r\le k) \Rightarrow r = a - kq$, где $a\in H$ и $kq\in H$, а значит,
    $r = 0$ и $H = k\Z$ $\blacksquare$

    \void
    \stmnt{Опр} Пусть $G$ - группа и $H$ - ее подгруппа. Путь фиксирован $g\in G$. Тогда левым смежным классом
    элемента $g$ по подгруппе $H$ называется множество:

    $$ gH = \{g\cdot h\vert h\in H\} $$

    Аналогично правым смежным классом является такое множество:
    $$ Hg = \{h\cdot g\vert h\in H\} $$

    \void
    \stmnt{Лемма 1} $\forall g_1, g_2\in G$ либо $g_1H = g_2H$, либо $g_1H\cap g_2H = \varnothing$.

    \void
    $\square$ Если $g_1H\cap g_2H\neq\varnothing$, то $\exists h_1,h_2\in H: g_1h_1 = g_2h_2\Rightarrow
    g_1 = g_2\cdot h_2\cdot h_1^{-1} \Rightarrow g_1H = g_2\cdot h_2\cdot h_1^{-1}H\subseteq H$.
    А так как $h_2\cdot h_1^{-1}\in H$, то $g_1H\subseteq g_2H$. Аналогично существует
    и обратное включение, а значит $g_1H = g_2H$. $\blacksquare$

    \void
    \stmnt{Лемма 2} $\abs{gH} = \abs{H}$ $\forall g\in G$ и для любой конечной подгруппы $H$.

    \void
    $\square$ $\abs{gH}\leq \abs{H}$. Если $gh_1 = gh_2 \Rightarrow g^{-1}gh_1 = g^{-1}gh_2
    \Rightarrow h_1 = h_2$, то есть совпадений нет. $\blacksquare$

    \void
    \stmnt{Опр} Индексом подгруппы $H$ в группе $G$ называется количество левых смежных классов $G$ по подгруппе $H$.

    Обозначение: $[G:H]$

    \void
    \stmnt{Теорема (Лагранжа)}

    Пусть $G$ - конечная группа, $H$ - ее подгруппа, тогда $\abs{G} = \abs{H}\cdot [G:H]$.

    \void
    $\square$ Любой элемент группы лежит в своем левом смежном классе по $H$, и смежные классы не пересекаются
    (по лемме 1) и любой из этих смежных классов содержит по $\abs{H}$ элементов (по лемме 2).
    $\blacksquare$

    \void
    \stmnt{Следствие 1} Пусть $G$ - конечная группа и взят элемент $g\in G$. Тогда $ord(g)$ делит $\abs{G}$.

    \void
    $\square$ Возьмем $H = \langle g\rangle$. Мы знаем, что $\abs{\langle g\rangle} = ord(g)$ и
    $\abs{G} = \abs{\langle g\rangle}\cdot [G:H]$, то есть $\abs{G}\vdots ord(g)$.
    $\blacksquare$

    \void
    \stmnt{Следствие 2} Пусть $G$ - конечная группа, тогда $g^{\abs{G}} = e$.

    \void
    $\square$ Применим следствие 1: $\abs{G} = ord(g)\cdot s \Rightarrow g^{\abs{G}} = g^{ord(g)\cdot s} =
    (g^{ord(g)})^s = e^s = e$.
    $\blacksquare$

    \void
    \stmnt{Следствие 3} aka Малая Теорема Ферма:

    Пусть $\overline{a}$ - ненулевой вычет попростому подулю $p$, тогда $\overline{a}^{p-1} = \overline{1}$,
    то есть $a^{p-1}\equiv 1$ (mod $p$).

    \void
    \textit{$\overline{0}, \overline{1},\dots,\overline{p-1}$ - вычеты по модулю $p$, то есть остатки от деления
    $m\in\Z$ на $p$.}

    \void
    $\square$ На самом деле это следствие 2 ($g^{\abs{G}} = e$), примененное к группе 
    $\Z_p^* = \{ \Z_p\backslash \{0\}, \cdot \}$, где $Z_p$ - множество всех вычетов по модулю $p$.
    $\abs{Z_p^*} = p - 1 \Rightarrow \overline{a}^{\abs{Z_p^*}} = e$.
    $\blacksquare$

    \void
    \stmnt{Зам} Точно так же можно было рассмотреть и правые смежные классы. Но число левых смежных классов
    равно числу правых и равно $\frac{\abs{G}}{\abs{H}}$ (по теореме Лагранжа).
 
\end{document}
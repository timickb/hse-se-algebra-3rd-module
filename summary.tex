\documentclass[a4paper,11pt]{article}

\usepackage[utf8]{inputenc}
\usepackage[english,russian]{babel}
\usepackage{titling}
\usepackage{titlesec}
\usepackage{amsfonts,amsmath,amssymb,amsthm,mathtools}
\usepackage{icomma}
\usepackage{cmap}
\usepackage{mathtext}

\titleformat{\section}
  {\normalfont\normalsize}{\fbox{}}{1em}{}

\newcommand{\stmnt}[1]{\fbox{\textbf{#1}}}
\newcommand{\void}{\vspace{\baselineskip}}
\newcommand{\R}{\mathbb{R}}
\newcommand{\N}{\mathbb{N}}
\newcommand{\Z}{\mathbb{Z}}
\newcommand{\abs}[1]{\vert #1\vert}

\setlength{\parindent}{0em}
\setlength{\droptitle}{-7em}

\usepackage[left=2cm,right=2cm,
    top=2cm,bottom=2cm,bindingoffset=0cm]{geometry}

\title{Алгебра на ФКН ПИ}
\author{Общий конспект всех лекций за 3 модуль}

\begin{document}
    \maketitle

    \stmnt{Утв} Пусть $G$ - группа и $g\in G$.

    Тогда $\abs{<g>} = ord(g)$, где $\abs{<g>}$ - число элементов в циклической группе, порожденной элементом $g$.

    \void
    $\square$ Заметим, что если $g^k = g^s \Rightarrow g^{k-s} = e$ (так как $\exists g^{-1}$) $\Rightarrow$
    порядок $g \leq k - s \Rightarrow$ если $g$ имеет бесконечный порядок, то все элементы $g^n$ $(n\in\Z)$
    различны и $\Rightarrow <g>$ содержит бесконечно много элементов. В бесконечном случае доказано.
    
    \void
    Если же $ord(g) = m$, то из минимальности $m\in\N \Rightarrow e=g^0, g=g^1, g^2,\dots,g^{m-1}$ попарно
    различны. Покажем, что $<g> = \{e, g, g^2,\dots,g^{m-1}\}$:

    $\forall n\in\Z: n = m\cdot q + r$, где $0\leq r \le m \Rightarrow g^n = g^{mq+r} = (g^m)^q\cdot g^r =
    e^q\cdot g^r = g^r$, где $0\leq r\le m \Rightarrow <g> = \{e,g,g^2,\dots,g^{m-1}\}$
    и $\abs{<g>} = m = ord(g)$. $\blacksquare$

    \void
    \stmnt{Утв} Пусть $f: G\implies F$ - гомоморфизм. Тогда $f$ - инъективен (т.е. является мономорфизмом)
    $\Leftrightarrow \ker{f} = e_G$, где $e_G$ - нейтральный элемент в группе в $G$, а $\ker{f}$ в данном
    случае является тривиальным \textbf{ядром} гомоморфизма.

    \void
    \stmnt{Опр} Ядром гомоморфизма $f: G\implies F$ называется множество элементов группы G, которые переходят
    в $e_F$ - нейтральный элемент во второй группе.

    $$\ker{f} = \{g\in G\vert f(g) = e_F\}$$

    \void
    \stmnt{Зам} $\ker{f}$ никогда не является пустым множеством, так как по свойству гомоморфизма $f(e_G) = e_F$.

    \void
    $\square$

    \fbox{$\Rightarrow$} Необходимость:

    Дано: $\forall x_1 \neq x_2: f(x_1)\neq f(x_2)\Rightarrow f(e_G) = e_F$ 
    (и для $x\in G$ $(x\neq e_G)$ $f(x)\neq f(e_G) = e_F$).

    \fbox{$\Leftarrow$} Достаточность:

    Дано: $\ker{f} = e_G$. Допустим, что $\exists x_1\neq x_2: f(x_1) = f(x_2)$.
    Тогда $f(x_1\cdot x_2^{-1}) = f(x_1)\cdot f(x_2^{-1}) = f(x_1)\cdot(f(x_2))^{-1} = e_F$
    $\Rightarrow x_1\cdot x_2^{-1} = e_G \Leftrightarrow x_1 = x_2$ - противоречие, значит,
    $f$ инъективно. $\blacksquare$

\end{document}